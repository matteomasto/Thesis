\documentclass[12pt]{book}
\usepackage{times}
% \usepackage{graphicx}
% to change the style of the bibliography, you might can check out file such as:
% /usr/share/texlive/texmf-dist/tex/latex/biblatex-nature/nature.bbx
% set doi to true for instance
\usepackage[backend=biber, style=nature]{biblatex}
\addbibresource{MyLibrary.bib}

\usepackage{booktabs}
\usepackage{listings}
\usepackage{array}
\setlength{\topmargin}{-0.5in}
\setlength{\textheight}{9in}
\setlength{\oddsidemargin}{0in}
\setlength{\textwidth}{6.5in}
\usepackage{longtable}
\usepackage{float}      
% Configure listings for a smaller monospace font
\lstset{
    basicstyle=\scriptsize\ttfamily,
    breaklines=true,
    columns=fullflexible,
    showstringspaces=false,
    frame=none
}
\usepackage[english]{babel}
\usepackage[utf8]{inputenc}
\DeclareUnicodeCharacter{2212}{-}
\usepackage[T1]{fontenc}
\usepackage[a4paper,left=3cm,right=2.5cm,top=3cm,bottom=3cm, twoside]{geometry}
\usepackage{xcolor}
\usepackage{stmaryrd}
\usepackage{amssymb}
\usepackage{amsfonts} 
\usepackage{nicefrac}
\usepackage{microtype}
\usepackage{amsmath}
\usepackage{mathtools}
\usepackage{libertine} % the font!
\usepackage[pdftex]{graphicx}
\usepackage{caption}
\usepackage{subcaption}
\usepackage{float}
\usepackage{multirow}
\usepackage[font=small,labelfont=bf]{caption}
\usepackage{apalike}
\usepackage{csquotes}
% \usepackage{minitoc}
\usepackage{braket}
\usepackage{tikz}
\usepackage{wrapfig}
\usepackage{enumitem}
\usepackage{epigraph}
\usepackage{longtable}
\usepackage{tikz}
\usepackage{siunitx}

\sisetup{
    detect-all,
    separate-uncertainty = true,
    multi-part-units = single
}

\tikzset{basic/.style={draw,fill=none,
                       text badly centered,minimum width=3em}}
\tikzset{input/.style={basic,circle,minimum width=3.5em}}
\tikzset{weights/.style={basic,rectangle,minimum width=2em}}
\tikzset{functions/.style={basic,circle, minimum width=4em}}
\newcommand{\addaxes}{\draw (0em,1em) -- (0em,-1em)
                            (-1em,0em) -- (1em,0em);}
\newcommand{\relu}{\draw[line width=1.5pt] (-1em,0) -- (0,0)
                                (0,0) -- (0.75em,0.75em);}
\newcommand{\stepfunc}{\draw[line width=1.5pt] (0.65em,0.65em) -- (0,0.65em) 
                                    -- (0,-0.65em) -- (-0.65em,-0.65em);}
\usetikzlibrary{arrows,matrix,positioning,shapes,arrows}
\usetikzlibrary{shapes.geometric, arrows, calc, intersections}
\newcommand{\tikznode}[2]{\relax
    \ifmmode%
    \tikz[remember picture,baseline=(#1.base),inner sep=0pt] \node (#1) {$#2$};
    \else
    \tikz[remember picture,baseline=(#1.base),inner sep=0pt] \node (#1) {#2};%
    \fi}
    
\usepackage{amsthm}
\theoremstyle{remark} 
\newtheorem{remark}{Remark} 
\usepackage[linesnumbered,ruled,vlined]{algorithm2e}
\SetKwInput{KwInput}{Input}                % Set the Input
\SetKwInput{KwOutput}{Output}              % set the Output
\usepackage{lscape}
\usepackage{titletoc}
\usepackage{algorithmic}
%\usepackage{cite}
\usepackage{booktabs}
\usepackage{arydshln}
\usepackage[nonumberlist]{glossaries}
\usepackage{calc}
\usepackage[]{titlesec} 

\definecolor{linkColor}{HTML}{32a852}
\usepackage[colorlinks=true,citecolor=linkColor,linkcolor=black]{hyperref}
\urlstyle{same}

\usepackage{fancyhdr}
\setcounter{tocdepth}{1}
\def\erf{\text{erf}}
\newcommand{\eqdef}{=\mathrel{\mathop:}}
    \usepackage[toc,page,title,titletoc,header]{appendix} % pour les annexes
    \renewcommand{\appendixtocname}{Table des annexes} % indique le nom de la table des annexes dans la toc
    \renewcommand{\appendixpagename}{Annexes} % Nom du titre de la page des annexes
    \usepackage{etoolbox}
    \AtBeginEnvironment{appendices}{\renewcommand{\thesection}{\Alph{section}}}
%%%%%%%%%%%%%%%%%%%%%%%%%%%%%%%%%%%%%%%%%%%%%%%%%%%%%%%%%%%%%%%%%%%%%%%%%%
%%%%%%%%%%%%%%%%%%%%%%%%%%%%%%%%%%%%%%%%%%%%%%%%%%%%%%%%%%%%%%%%%%%%%%%%%%

\definecolor{yourcolor}{HTML}{003bb2}%{008bb2}, 8a0e19

\titleformat{\chapter}[display]
{\normalfont\color{yourcolor}}
{\filleft\huge\color{black}\textsc\chaptertitlename\hspace*{2mm}%
	\begin{tikzpicture}[baseline={([yshift=-.6ex]current bounding box.center)}]
                	\node[fill=yourcolor,circle,text=white] {\thechapter};
	\end{tikzpicture}
}
{1ex}
{\titlerule[1.5pt]\vspace*{1.5ex}\Huge\color{black}\textsc}
[]

\titleformat{name=\chapter, numberless}[display]
{\normalfont\color{black}}
{}
{1ex}
{\vspace*{-5cm}\Huge\textsc}
[]

%command to print the actual minitoc
\newcommand{\printmyminitoc}[1]{%
	\noindent\hspace{1cm}%
	\colorlet{chpnumbercolor}{black}%
	\begin{tikzpicture}
	\node(s){
		\begin{minipage}{.9\linewidth}%minipage trick
		\printcontents[chapters]{}{1}{}
		\end{minipage}
	};
	{
		\color{yourcolor}
		\draw(s.north west)--(s.north east) (s.south west)--(s.south east);
	}
	\end{tikzpicture}
	\vspace*{3ex}
	
	#1
	\vfill
	\pagebreak
}

\providecommand{\keywords}[1]{\textbf{\textit{Keywords---}} #1}
\DeclareMathOperator*{\argmin}{arg\,min}
\DeclareMathOperator*{\argmax}{arg\,max}

\newcommand{\HRule}{\rule{\linewidth}{0.7mm}}
\newcommand{\Hrule}{\rule{\linewidth}{0.3mm}}

\let\oldclearpage\clearpage
\def\clearpage{%
	\oldclearpage%
	\thispagestyle{empty}%
}



% hack to hyperlink doi to journal title - https://tex.stackexchange.com/questions/23832/biblatex-make-title-hyperlink-to-doi-url-if-available
\newbibmacro{string+doi}[1]{\iffieldundef{doi}{#1}{\href{http://dx.doi.org/\thefield{doi}}{#1}}}
\DeclareFieldFormat*{journaltitle}{\usebibmacro{string+doi}{\emph{#1}}}
% \DeclareFieldFormat*{volume}{\usebibmacro{string+doi}{\textbf{#1}}}
\DeclareFieldFormat*{pages}{\usebibmacro{string+doi}{#1}}



%%%%%%%%%%%%%%%%%%%%%%%%%%%%%%%%%%%%%%%%%%%%%%%%%%%%%%%%%%%%%%%%%%%%%%%%%%%%%%%

\begin{document}
\begin{titlepage}
	
\begin{figure}
     \center
    \includegraphics[height=3cm]{figures/logos/full_logo.png}
\end{figure}
\hrule
		
\vfill

\begin{center}	
	{
        \Large \bfseries Deep Learning for Bragg Coherent Diffraction Imaging: Detector Gap Inpainting and Phase Retrieval
    }
\end{center}	
		
\vfill
\begin{center}	
	{\Huge  Thesis}
	\vfill
	présentée et soutenue publiquement le %{\Large TBD}
	\vfill
	Pour l'obtention du titre de 
	\vfill
	
	{\Large Docteur de l'Université Grenoble Alpes}
	
	(mention Physique du rayonnement et de la matière condensée)
	
	\vfill
	par 
	
	Matteo Masto
	
\vfill
sous la direction de 

Dr. Tobias Sch\"ulli, Dr. Vincent Favre-Nicolin, Dr. Steven Leake
	
\end{center}


\vspace{1cm}
			
{\bfseries Composition du Jury}		
\begin{center}
	\begin{tabular}{lll}
		%FULLNAME & Grade & \textbf{Président de jury}\\
		
		XXXXXXXXXX &  PR, XXXXXXXXXX & \textbf{Rapporteur}\\
		
		XXXXXXXXXX  &  PR, XXXXXXXXXX & \textbf{Rapporteur} \\
		
		XXXXXXXXXX & PR, XXXXXXXXXX & \textbf{Examinateur} \\
		
		XXXXXXXXXX & PR, XXXXXXXXXX & \textbf{Examinateur} \\
		
		Tobias Sch\"ulli &  ESRF & \textbf{Directeur de Thèse}\\
		Vincent Favre-Nicolin &  ESRF UGA, & \textbf{Directeur de Thèse}\\
		Steven Leake &  ESRF & \textbf{Directeur de Thèse}

	\end{tabular}\\[1cm]
\end{center}
\hrule

{\small \bfseries
École Doctorale de Physique-Grenoble}

\newpage
\end{titlepage}



\pagestyle{fancy}

\fancyhead{}

\renewcommand{\chaptermark}[1]{\markboth{\textsc{#1}}{}}

\frontmatter

%%% *********************************************************************
% Corps "formel" de la thèse : table des matières, liste des tableaux, etc.
%%% *********************************************************************


%\input{Chapters/Remerciements.tex}
%\input{Chapters/Summary.tex}


% Table des matières
\tableofcontents
% \addcontentsline{toc}{chapter}{Table des matières}
\clearpage

% Liste des figures
%\listoffigures
%\addcontentsline{toc}{chapter}{Liste des figures}
%\clearpage


% Liste des tableaux
%\listoftables
%\addcontentsline{toc}{chapter}{Liste des tableaux}
%\clearpage


% Spacing de ligne de nouveau normal
\setlength{\parskip}{.7em}

\titlespacing*{\section}{0pt}{.9em}{.8em}
\renewcommand{\baselinestretch}{1.1}



\newgeometry{
    %  total={170mm,257mm},
    width=160.02mm,
    % height=11.69in,
    % right=0.79in,
    left=1.18in,
    % top=12.446mm,
	top=20.446mm,
    % bottom=20.066mm,
	bottom=30.066mm,
 }
 
\mainmatter

% Introduction dans un fichier intro.tex dans le dossier Chapters
% \part{Preface}

% The textheight and textwidth across the document are \the\textheight and \the\textwidth respectively.

\fancyhead[RO]{\leftmark}
\fancyhead[LE]{\textsc{\chaptername~\thechapter}}

% CHAPTER 1 - Introduction
\chapter{Introduction}
\label{chap:introduction}
\section{Introduction}\label{chp:intro}

In this manuscript, the use of Deep Learning methods, and more in general of GPU accelerated optimizations, for the advance of 
the data analysis in Bragg Coherent Diffraction Imaging (BCDI) will be presented. However, before delving into the 
study’s developments, I would like to share with the reader a reflection that has taken shape over the course of this 
PhD, serving as a kind of preface. In particular, I have come to observe that, unlike other more fundamental 
scientific investigations, this work originates from the practical limitations of the technique in question. 
It is indeed because the detectors are imperfect—unable to record flawless images due to gaps, or incapable of 
capturing phase information because its oscillations are too rapid—that one is compelled to manipulate the available 
data with sophisticated algorithms. And, as often happens in science, compensating for these technical shortcomings 
leads to the development of tools rooted in the most abstract realms of mathematics and information theory. How much 
missing information can one extract from a signal? How can it be extracted, and under what conditions? In which 
circumstances is it easier, and why? Thus, a fascinating world opens up not when we directly investigate the foundations 
of matter, but when we examine \textit{how} we go about investigating them. 

Although this manuscript is ultimately focused on the specific cases of BCDI gap inpainting and phase retrieval, 
I hope to convey at least a bit of the wonder and awe that comes from knowing that such applications draw their roots 
from far deeper, more general, complex, and abstract themes.

\subsection{PhD objectives}

As we will soon present in details, we can anticipate that BCDI is a powerful imaging-microscopy technique performed at synchrotron 
and XFEL facilities and for its non-invasive, high spatial resolution, investigation of the atomic structure of single crystal 
nano-particles it has already been proved successful in many diverse fields. In fact, the study of the internal 
strain distribution, defects population and morphology at the nanometer scale (typical BCDI resolution is of the order of 10 nm) 
under various physio-chemical environments is of crucial importance for fundamental science as well as for engineering 
applications \cite{BCDI_review2024}. Since the very first use in 2001 by Prof. Ian Robinson and coauthors \cite{Robinson_gold_2001} for the imaging 
of gold particles, BCDI has been employed for the analysis of materials of relevance for nanotechnology and electronics 
\cite{Favre-Nicolin_2010} as well as for Li-ion and Na-ion batteries \cite{Singer2018, Serban2024}, catalysis 
\cite{atlan_imaging_2023} and recently on biological materials too \cite{Grunewald:ro5042}. 

However, this technique strongly relies on computer algorithms for the transformation of the acquired diffraction pattern into 
the real space particle shape and strain field. For this reason, in the past years, numerous efforts have been made  
to make the computational aspect of BCDI robust and reliable. Some of these developments will be presented later on, while 
here we want to highlight that the research in this field has become even more active and prosperous with the advent of machine 
learning (ML). Moreover, the recent upgrade to fourth-generation x-ray light source of many synchrotrons across the planet 
(MAX-IV - Sweden in 2017, ESRF-EBS - France and Sirius - Brazil in 2020 and others scheduled for the coming years) is 
boosting the crystalline nano-imaging techniques such as Bragg CDI and Bragg ptychography \cite{Li2022, Leake:il5024}. 
Hence, the need for fast and optimized ways to handle the large amounts of experimental data. It is exactly in this context 
that this study is framed, with the initial goal of exploring the advantages that machine learning algorithms can bring 
to the BCDI technique. \\
Specifically, the work has focused on two main tasks of primary importance for the reliability of the processed data. 
As briefly mentioned above and discussed more in details later, the detectors employed for x-ray imaging techniques 
have the main limitation of not being capable of record the phase of the impinging x-ray light, thus leaving its retrieval 
to computer algorithms. Moreover, these detectors are built with some non sensing 




% CHAPTER 2 - Bragg Coherent Diffraction Imaging
\chapter{Bragg Coherent Diffraction Imaging}
\label{chap:bcdi}

In this chapter some basic theoretical insights about the BCDI technique are provided, with the aim to highlight the key 
concepts, assumptions and physical interpretations. More thorough descriptions can be found in papers, textbooks 
and PhD manuscripts. I will adopt the formalism of Als-Nielsen and McMorrow in \cite{alsnielsen_mcmorrow2011} but similar 
derivations and complementing observations can be found in \cite{guinier1994, paganin2006coherent} as well as some more recent papers \cite{vartanyants2013coherentxraydiffractionimaging} 
and PhD thesis \cite{dupraz:tel-01285735, girard:tel-02906931}
, 

\section{Foreword on typical assumptions and approximations in BCDI}
In order to keep the dissertation short and targeted to the BCDI case, I will start considering some observations on this
technique that will lead to some preliminary assumptions and simplifications. First, the word \textit{``Bragg''} suggests that 
crystalline specimens are involved. As discussed later in the text, Bragg's law applies to periodic structures, therefore 
we will limit our discussion to this specific case. \\
The word \textit{``Coherent''} implies that samples are probed with coherent beams (in our case X-rays). This fundamental property of 
synchrotron radiation will be briefly discussed later on. For the moment, this ingredient enables us to express the 
probing radiation with plane electromagnetic waves. \\
The word \textit{``Diffraction''} refers to the type of mechanism describing the interaction between the X-rays and the 
samples. Paraphrasing \cite{guinier1994} at page 4, this mechanism can be divided into two main phenomena, namely (i) 
the scattering of the radiation by each individual atom in the sample and (ii) the interference between the waves scattered 
by these atoms. The interference mechanism, in turn, is enabled because these scattered waves are coherent with the incident 
radiation and therefore between themselves. In other words, the information of each scatterer is shared with the other scatterers 
as the diffracted waves ``talk to each other''. The complete mathematical description of these two phenomena without approximations 
is prohibitive, hence some simplifications usually adopted: 
\begin{itemize}
    \item \textbf{No refraction, no absorption}: Scattering is the only mechanism considered. Because of their short wavelengths
    (0.5 - 2.5 \AA), X-rays are practically never deviated by refraction. Moreover, we assume to always operate at energies that 
    are far from absorption edges of the probed materials, thus neglecting any absorption effect. 
    \item \textbf{Elastic scattering}: The interaction between the incoming X-rays and the atom is considered only elastic, meaning that 
    no energy nor momentum is transferred to the atom, which bounces off instead the photons with unaltered energy and 
    momentum. This is again necessary for the scattered waves to interfere, as any difference in wavelength would 
    prevent any coherent interaction. This description, also called Thomson scattering, considers the interaction with a free charge, 
    and it also shows that cross-section of the scattering of electrons is much higher than the one of protons. 
    For this reason, only the scattering from electrons is considered. 
    \item \textbf{Weak diffraction (Born approximation)}: This assumption implies that each scattered wave does not interact 
    further with the sample, therefore neglecting any possible multiple scattering event. The consequence of this assumption 
    is that the overall diffracted wave can be approximated by the linear superposition of the contributions of each scattering site. 
    This approximation, in crystallography, is called \textit{kinematical approximation}. 
    Dealing with crystalline samples, this assumption breaks for relatively thick samples ( $ > 1 \mu m $) in which the 
    light travels through the sample for longer distances before exiting, therefore bouncing off several atoms. 
    Diffraction of larger samples requires more complex theory of the so-called \textit{dynamical regime}.
    However, in our case, the size of the typical samples studied with BCDI hardly exceeds $ 1 \mu m $ size, making the 
    kinematical approximation suitable. 
    % \item \textbf{Projection approximation}: According to Paganin \cite{paganin2006coherent}
    \item \textbf{Far-field approximation}: Here, the distance between the scattering atoms and the detector is assumed 
    to be much larger than the distance among the scatterers themselves. One can intuitively see that this approximation 
    turns the spherical waves created by the scatterers, interfering with each other, into plane waves when these are evaluated 
    far from the sources (in this case the atoms). This assumption is always respected in the BCDI 
    technique as the sample-detector distance is in the order of tens of centimeters.

    % \item \textbf{Linear Polarization}: While not usual in textbooks, here, for simplicity we will consider linearly 
    % polarized x-rays. This restriction 

\end{itemize}

The last word \textit{``Imaging''} tells us that the format of the data is by nature, multidimensional (2D - 3D). It 
will be shown later in the chapter that 3D diffracted signal is recorded stacking 2D images captured by the detector, 
and therefore the results after the data analysis are 3D images of the sample.

Given this set of assumptions and approximations we can proceed with our simplified derivation of the equation governing 
the coherent X-ray scattering from a crystal and its interpretation. 

\section{Coherent X-ray scattering from crystalline structures}

Let us consider an X-ray beam, represented by a perfectly monochromatic plane wave with linear polarization in the horizontal 
plane, scattering with a single free electron. In this simple case we can imagine the X-ray electromagnetic field exerting 
a force onto the electron placed in the origin. In turn, this force will accelerate the electron accordingly, therefore inducing  
an electromagnetic wave as well. Being the scattering assumed to be elastic, the radiation produced by the oscillating 
electron (\textit{electric dipole approximation}) will have the same wave-vector of the incoming X-ray. Moreover, the 
solution of Maxwell equations for this specific case 
shows that this dipole radiation propagates in the form of a spherical wave. At this point we ask ourselves what is the amplitude 
of this scattered wave when evaluated in a generic point $\mathbf r$ on the vertical plane, far from the origin (\textit{far-field approximation}). 
The result was achieved by Thomson in 1906 and is here reported without the full detailed derivation which can be found in the 
cited textbooks. 

% \begin{equation}
%     \mathbf{E}_{\text{dip}}(\mathbf{r},t) 
%     = r_e \, \frac{e^{ikr}}{r} \, e^{-i \omega t} 
%     \left[ \hat{r} \times \left( \hat{r} \times \mathbf{E}_0 \right) \right] 
%     e^{-i \mathbf{Q} \cdot \mathbf{r}'} 
%     \label{eq:scattering_pointlike}
% \end{equation}

\begin{equation}
    \mathbf{E}_{\text{dip}}(\mathbf{r},t) 
    = -r_0 \, \frac{e^{ikr}}{r} \, e^{-i \omega t} 
    E_0 \mathbf{\hat{x}}
    \label{eq:scattering_pointlike}
\end{equation}

where $r_0$ is the classical radius of the electron, or Thomson scattering length, $k$ is the outgoing wave-vector, $\omega$ is the pulsation of the 
X-ray beam (incoming and outgoing), $\mathbf{E}_0$ is the electric field of the incoming radiation.

In this case we cannot talk about diffraction as there is no interference of the outgoing wave with other scattered waves. In order 
to have a diffraction pattern we need to have at least a second charge scattering, from which a phase 
delay with respect to the first one can be calculated. For instance, if we consider $N$ electrons, separated in space 
by a distance $\mathbf{r'}$ we could evaluate the contribution to the overall scattering wave-field for each electron. 
The simplest way is to make use of the \textit{kinematical approximation} and sum linearly all the contributions. However, we must 
take into account the phase delays between the scattering from different positions in space. 
This phase delay can be calculated and it turns out to be $\Delta\phi(r) = (\mathbf k - \mathbf {k'})\cdot \mathbf r = \mathbf Q \cdot \mathbf r$ 
where we have expressed the difference between the wave vectors with $\mathbf{Q}$ often called \textit{scattering vector}.
The equation can thus be rewritten like: 

\begin{equation}
    \mathbf{E}_{\text{atom}}(\mathbf{r},t) 
    = r_e \, \frac{e^{ikr}}{r} \, e^{-i \omega t} 
    E_0 \mathbf{\hat{x}}
    \sum_{i = 1}^{N} e^{i \mathbf{Q} \cdot \mathbf{r}'} 
    \label{eq:scattering_electrons}
\end{equation}

In the continuum limit, replacing the $N$ point-like charges with an overall electron density distribution $\rho(\mathbf r)$ 
the above equation takes the form: 

\begin{equation}
    \mathbf{E}_{\text{atom}}(\mathbf{r},t) 
    = r_e \, \frac{e^{ikr}}{r} \, e^{-i \omega t} 
    E_0 \mathbf{\hat{x}}
    \int_{\mathbb{R}^3} \rho_a(\mathbf r') e^{i \mathbf{Q} \cdot \mathbf{r}'}  d^3 \mathbf r'
    \label{eq:scattering_atom}
\end{equation}

\begin{figure}[H]
    \centering
    \includegraphics[width=\textwidth]{figures/Intro/scattering.pdf}
    \caption{Sketch of the scattering process evaluated in the far-field on the vertical plane for an electron density irradiated by a monochromatic X-ray beam 
    with linear polarization along the $\mathbf{\hat{x}}$ direction.}
    \label{fig:scattering}
\end{figure}

It is clear now that the information regarding the physical system of interest is embedded in the integral term. 
In fact, this is often called \textit{``form factor''} - $F(\mathbf Q)$ - and it plays an important role in the interpretation of the 
scattering equations. 

\begin{equation}
     F(\mathbf Q) = 
    \int_{\mathbb{R}^3} \rho(\mathbf r') e^{i \mathbf{Q} \cdot \mathbf{r}'}  d^3 \mathbf r'
    \label{eq:formfactor}
\end{equation}

This last term represents the Fourier transform of the electron density, and it is the main result of this paragraph as it 
links the charge distribution of the sample in real space with the quantity measured, in reciprocal space. 

To continue, we should bear in mind that X-ray photon counting detectors are sensitive to the time-averaged intensity 
of the signal as their time response is much slower than the oscillating frequency of X-rays ($\sim 10^{9}$ Hz for typical 
read-out limited frame rates of the Maxipix 
\cite{ponchut_maxipix_2011} against the $\sim 10^{18}$ Hz for X-rays at 10 keV). This limitation is also at the core of 
the ``Phase Problem'' that we will see later on, for which the phase information of the complex-valued wave-field is 
lost in the measurement. 
In order to do so, the time-averaged Poynting vector is calculated. 

\begin{equation}
    \langle \mathbf{S(r)} \rangle = r_e^2 \frac{1}{r^2}J_0
    \left| \int \rho(\mathbf{r'}) e^{i \mathbf{Q}\cdot \mathbf{r'}} d^3 r' \right|^2 \mathbf{\hat{r}}
    \label{eq:poynting}
\end{equation}

where $J_0 = \left| \mathbf{E}_0 \right|^2 / 2\mu_0c$ is the incident intensity. 
To conclude we consider the power delivered on the detector. For a pixel with area $d\mathbf a = r^2d\Omega\mathbf{\hat{r}}$ 
the radiation power is equal to: 

\begin{equation}
    P(\mathbf{Q})= r_e^2 J_0
    \left| \int \rho(\mathbf{r'}) e^{i \mathbf{Q}\cdot \mathbf{r'}} d^3 r' \right|^2 d\Omega
    \label{eq:power}
\end{equation}

Eq.\ref{eq:power} shows that the signal captured by the detectors is now in $\mathbf{Q}$ space, and it is proportional 
to the square modulus of the Fourier transform of the electron density of the sample. The square modulus operation also 
shows how the phase of the Fourier transformed scattering amplitude is lost.

\subsection{One atom}

If now we were to consider an atom, far from resonance, we could assume the electron density being the main responsible 
for the scattering. It is known indeed that protons, because of the larger mass, have a much smaller cross-section for the
scattering with photons. 
Using Eq.\ref{eq:formfactor} we would therefore have the \textit{atomic form factor} - $f_l(\mathbf{Q})$ being defined as: 

\begin{equation}
    f_l(\mathbf{Q} = 
   \int_{\mathbb{R}^3} \rho_l(\mathbf r) e^{i \mathbf{Q} \cdot \mathbf{r}}  d^3 \mathbf r
   \label{eq:atomformfactor}
\end{equation} 

We now need to study the specific case in which the collection of atoms is ordered into a periodic structure. 

\subsection{Ensemble of ordered atoms: a Crystal}

Perfect crystals are constructed by a basic structural arrangement of atoms (\textit{motif}) repeated periodically on a 
\textit{lattice} of one or more dimensions. The regularity of the lattice is such that, for the 3D case, any of its nodes 
can be located in space by the formula: 

\begin{equation}
   \mathbf {R_n} = n_1\mathbf{a_1} + n_2\mathbf{a_2} + n_3\mathbf{a_3}
   \label{eq:lattice}
\end{equation}

where ${\mathbf{a_1},\mathbf{a_2},\mathbf{a_3}}$ constitutes the basis vectors of the primitive unit cell and $n_1, n_2, n_3$ are integer numbers. 
It follows that the information can be condensed in the unit cell, i.e. the orientation in space of the basis vectors as any region 
of the lattice can be seen as the same unit cell, translated from the origin by the amount given by the $\sqrt{n_1^2 + n_2^2 + n_3^2}$. \\

The overall crystal is then constructed positioning on each of the nodes of the lattice the same motif, or basis. 

In another more elegant way, we could say that, being the lattice $\mathcal{L}(\mathbf r)$, the basis $\mathcal{B}(\mathbf r)$, the crystal 
$\mathcal{C}(\mathbf r)$ is given by the convolution of $\mathcal{L}(\mathbf r)$ with $\mathcal{B}(\mathbf r)$ : 

\begin{equation}
    \mathcal{C}(\mathbf r)  = \mathcal{L}(\mathbf r) \ast  \mathcal{B}(\mathbf r)
    \label{eq:conv}
 \end{equation}

 For simplicity, we will consider from now on a single atom basis.
 At this point, when evaluating the scattering amplitude of the crystal we have to deal to an assembly of atoms, and we 
 may want to exploit the regular structure we have just described. 
 First, we can assume that, similarly to the case of many scattering electrons, in the kinematical approximation the overall 
 scattering factor is given by the sum of the contributions of each atom, weighted by a phase factor that accounts for their 
 positions in space. 

 \begin{equation}
    F_{crystal}(\mathbf{Q}) = 
   \sum_{l=1}^{\text{All atoms}} f_l(\mathbf Q) e^{i \mathbf{Q} \cdot \mathbf{r_l}} 
   \label{eq:crystalformfactor}
\end{equation} 

Secondly, observing that the position of each atom is given by the sum of the position of the atom inside the unit cell 
and the lattice vector $\mathbf{r_l} = \mathbf{R_n} + \mathbf{r_j}$, we can separate Eq.\ref{eq:crystalformfactor} in two terms: 

\begin{equation}
    F_{crystal}(\mathbf{Q}) = 
   \sum_{\mathbf{R_n} + \mathbf{r_j}}^{\text{All atoms}} f_l(\mathbf Q) e^{i \mathbf{Q} \cdot (\mathbf{R_n} + \mathbf{r_j})} = 
    \underbrace{\sum_{n} e^{i \mathbf{Q} \cdot \mathbf{R}_n}}_{\text{Lattice}}
    \underbrace{\sum_{j} f_j(\mathbf{Q}) e^{i \mathbf{Q} \cdot \mathbf{r}_j}}_{\text{Unit cell}}
   \label{eq:crystalformfactor2}
\end{equation} 

The first summation extends over all lattice points, while the second covers all atoms within the unit cell. At this 
stage, we can once again take advantage of the lattice periodicity to evaluate the large summation over all lattice 
points. 

\subsection{Laue condition and Bragg's Law} 

The term we want to calculate is the sum of complex exponential, meaning that if the phases $\mathbf{Q} \cdot \mathbf{R_n}$ 
are misaligned the sum will be \textit{incoherent} and the resultant will be very small, in the order of unity. 
On the contrary, when phase offsets are equal to an integer multiple of $2\pi$ the \textit{phasors} will add \textit{coherently} 
The problem is thus to find those $\mathbf{Q}$ values for which

\begin{equation}
   \mathbf{Q} \cdot \mathbf{R_n} = 2\pi \times \text{integer}
   \label{eq:laue}
\end{equation}

In order to do than we need to construct a reciprocal space lattice with a set of basis ${\mathbf{a_1^\ast}, \mathbf{a_2^\ast}, \mathbf{a_3^\ast}}$ 
which fulfill: 

\begin{equation}
    \mathbf{a_1} \cdot \mathbf{a_1^\ast} = 2\pi h \qquad \mathbf{a_2} \cdot \mathbf{a_2^\ast} = 2\pi k \qquad \mathbf{a_3} \cdot \mathbf{a_3^\ast} = 2\pi l 
   \label{eq:miller}
\end{equation}

where $ h, k, l $ known as Miller indices, are integer. Having a set of basis vectors and the Miller indices, the 
resulting reciprocal space lattice lies in those points found by the vector $\mathbf{G}$ 

\begin{equation}
    \mathbf{G} =  h\mathbf{a_1^\ast} + k\mathbf{a_2^\ast} + l\mathbf{a_3^\ast} 
   \label{eq:G}
\end{equation}

This result is telling us that the scattering amplitude of a diffracting crystal is detectable only in those points in 
space for which the wave-vector $\mathbf{Q}$ coincides with a point of the reciprocal lattice, hence $\mathbf{Q} = \mathbf{G}$. 

This is known as the Laue condition for diffraction as it was discovered by Max von Laue in 1912 \cite{FriedrichKnippingLaue1912}.\\ 
A different but equivalent interpretation of the diffraction of a crystal was given by William Lawrence Bragg in 1913 \cite{Bragg1913}. 
Here, the crystal lattice is seen as a stack of parallel planes and the condition for constructive interference of 
the waves scattered by planes of the same family is found as follows.
Let us consider an X-ray beam of wavelength $\lambda$ and propagation vector $\mathbf{k_i} $ impinging with an angle $\theta$ 
on a crystal. We call $d$ the distance between the planes of the crystal. The scattered beam is leaving the crystal with the 
same angle $\theta$ and with a propagation vector $\mathbf{k_f}$ equal in magnitude to the incident one (\textit{elastic scattering}). 
At this point one can find the relationship between $ \theta, d, \lambda$ that allows for a constructive interference of the 
waves diffracted from the series of planes by evaluating the optical path length difference induced by the spacing. 
Reminding that $\left|k_i\right| = \left|k_f\right| = 2\pi/\lambda $ and with the help of Fig. \ref{fig:bragg} we can observe that this 
difference is $\Delta l = 2 d \sin(\theta)$ and therefore the phase offset between two waves is $\Delta \phi = \left|k\right| \Delta l 
= 4 d\pi \sin(\theta)/\lambda $. We have seen above that the condition for constructive interference requires the phase 
differences to be equal to a multiple of $2\pi$, therefore: 

\begin{equation}
    2 d \sin(\theta) = n \lambda 
   \label{eq:Bragg}
\end{equation}


Equation \ref{eq:Bragg} is known as Bragg's law, and it can be demonstrated to be equivalent the Laue condition, in scalar form. 
In particular, as illustrated in Fig.\ref{fig:bragg}, the scattering vector $\mathbf Q$ is equivalent to the vector G

\begin{figure}[H]
    \centering
    \includegraphics[width=\textwidth]{figures/Intro/bragg.pdf}
    \caption{Illustration of Bragg's law.}
    \label{fig:bragg}
\end{figure}


\subsection{Finite size crystals}

\subsection{Strained crystals}

\section{BCDI at ESRF - ID01}\label{chp:id01}

\subsection{Synchrotron radiation}

\subsection{Coherence}

\subsection{Ewald sphere and Rocking curves}

\subsection{Detectors}

% CHAPTER 3 - Convolutional Neural Networks
\chapter{Convolutional Neural Networks}
\label{chap:dl_theory}


In this chapter a short overview on the basic concepts of Convolutional Neural Networks (CNNs) will be given. 
The scope is to give the necessary background for the understanding of the structure and the motivations behind the 
Deep Learning (DL) models employed for the analysis of BCDI data, presented in the next chapters.  

For more comprehensive and exhaustive dissertations about Machine Learning (ML) and Artificial Neural Networks (ANNs)
the book of Goodfellow \cite{Goodfellow_2016} and the more recent from Prince \cite{prince2023understanding} are 
suggested to the reader. 

\section{Artificial Neural Networks (ANNs)}

ANNs are a type of machine learning algorithm inspired by the biological neuron structure. ANNs are generally composed of interconnected 
nodes where the signal is processed through operations with tunable parameters named weights 
and biases for multiplications and addition respectively. An important feature of each node is the \textit{activation function}, 
that introduces a non-linear operation and returns the node's output \cite{jagtap2022}. Several kinds of these 
activation functions exist and their use depends on the properties of each (bounds, derivatives, positivity, etc.) 
\cite{kunc2024}. In the following chapters the modified rectifying linear unit, known as LeakyReLU \cite{Maas2013RectifierNI}, 
and the sigmoid, also known as logistic, function will be used. Neurons are generally organized into \textit{layers} and 
are connected to neurons of other layers. In \textit{feed-forward} neural networks the information flows from the 
input layer to the output layer with forward connections only.
In other neural networks, like \textit{recurrent} ones, the connections are also designed backwards. 

\begin{figure}[H]
    \centering
    \includegraphics[width=\textwidth]{figures/Intro/neuron2.pdf}
    \caption{\textbf{Schematic of an artificial neuron and a neural network.} On the left, a series of input signals 
    $x_i$ are multiplied by the tunable weight $w_{k}$ and summed together with the tunable bias $b_k$ relative to 
    the $k$-th neuron. The output is then passed through the activation function $f$ which produce an output $y_k$ 
    which is a non-linear combination of the inputs. On the right, the feed-forward network composed 
    of two \textit{hidden} layers of artificial neurons processing the ten units long vector $x$ and returning a 
    binary output. Networks of this type are said to be \textit{fully connected} as each input component, or 
    node's output is processed by all neurons. Adapted from \cite{haykin2009neural}}
    \label{fig:neuron}
\end{figure}

\subsection{Neural networks as universal approximators}

The use of non-linear functions is found to be fundamental for the powerful analytical and statistical properties of ANNs, 
which have been progressively 
established over the years. It was shown \cite{Cybenko1989, Hornik_1989, Leshno_1993} that ANNs with appropriate 
activation functions and sufficient parameters can approximate \textit{any} continuous function on a compact domain to arbitrary 
accuracy (\textit{universal approximation theorems}). Later the proof has been extended to deep convolutional neural 
networks as well \cite{UniversalityCNN_2020}.

Formally, we can consider $\mathcal{X}$ as the set of all events belonging to the same statistical distribution and 
$\mathcal{Y}$ similarly. We also assume that an unknown mapping $\mathcal{M}$ between an input $x \in \mathcal{X}$ and the output 
$y = \mathcal{M}(x)$ exists. At this point, the goal of the ANNs is to be the closest approximation of $\mathcal{M}$.
Typically, this implies seeking the combination of parameters $\theta_k = (w_k, b_k)$ of the neural network $\mathcal{N}_{\theta}$ 
such that $\mathcal{N}_{\theta} \approx \mathcal{M}$

\textbf{How is this mapping found?} The core idea is known as Empirical Risk Minimization and states that 
when the $\theta_k$ are adjusted to fit a sufficiently large dataset consisting of samples drawn from an underlying distribution, 
the function being approximated reflects the statistical relationships encoded in that distribution \cite{Hornik_1990}.
By the Law of Large Numbers, the empirical distribution observed in the \textit{training set} converges to the true 
distribution as the sample size grows, and thus the empirical risk minimized during training approaches the true 
risk. This statistical foundation explains why ANNs are able to generalize to new, unseen data drawn from the 
same population. 

It is therefore sufficient to have a large but finite number of samples representative of both the sets $\mathcal{X}$ 
and $\mathcal{Y}$ to approximate $\mathcal{M}$. This powerful statistical property underlies the concept of \textit{supervised} 
training of ANNs. 

More formally, one can consider having a limited set of $N$ examples $(x_1,y_1 = \mathcal{M}(x_1)), ..., (x_N, y_N = \mathcal{M}(x_N))$ 
which we call \textit{training set} $\mathcal{T}$. Here, each $x_i$ is an input instance and $y_i$ the 
corresponding ground truth transformation operated by $\mathcal{M}$. We can now introduce a non-negative scalar 
real-valued \textit{loss function} $L(\hat{y},y )$ which measures the difference between the ground truth $y$ and the 
output of the neural network $\hat{y} = \mathcal{N}_\theta (x)$ for each element of the training set.

It follows that the best approximation of the mapping $ \mathcal{N}_\theta^{\mathcal{T}} \approx \mathcal{M}^{\mathcal{T}}$ is 
obtained when the score of the loss function averaged across the number of training samples is lowest.
The problem of approximating the mapping between the input and ground truth in the training set is then formulated as 
a minimization problem of the type: 

\begin{equation}
    \hat{\theta}_{\mathcal T}
    = \arg\min_{\theta} 
    \left\{ \frac{1}{N} \sum_{i=1}^N 
    L\!\left(y_i, \mathcal{N}_{\theta}^{\mathcal{T}}(x_i)\right) 
    \right\}
\end{equation}

For a sufficiently large $N$ the $\hat{\theta}_{\mathcal T} \approx \hat{\theta}$ valid for the whole statistical distribution. 
The size of $N$ needed to approach the true mapping depends obviously on the complexity of the mapping and on the 
complexity of the minimization task. While the first is inherent to the problem, the second can be engineered with 
the choice of a metric for which the loss function minimization is easier. This will be clear in Chapter \ref{chap:phase_retrieval}
where different loss functions are tested. In the same way, the network design can impact the facility with which the 
ANN converges to the approximation of the desired mapping. In this regard, the use of multiple intermediate layers 
of neurons (\textit{hidden layers}) was shown to strongly improve the ability of the NN to fit more complex functions
(see \cite{prince2023understanding} - Chapter 4.5). 
For this reason these network called \textit{deep} have taken over \textit{shallow} ones. Another example, briefly 
discussed in the next section, is given by Convolutional Neural Networks (CNNs), a type of ANN suited for natural image 
processing.  

Moreover, in cases in which one does not have a training set composed of input - ground truth pairs, but instead 
incorporates prior knowledge into the model architecture or the loss function, the training is said to be 
\textit{unsupervised}. This approach has not been explored in the context of this PhD. 

\subsection{Gradient Descent}

At this point another question may arise: \textbf{How do we solve the minimization problem?}
The most straightforward manner to solve a minimization problem is with gradient descent. This implies an iterative process 
in which at each iteration the derivatives of the loss function with respect to each parameter $\theta$ are calculated, 
and each parameter is updated correspondingly. 

\begin{equation}
    \theta^{t+1}
    = \theta^{t} - \eta \nabla_{\theta} \left[ \frac{1}{N} \sum_{i=1}^N 
    L\!\left(y_i, \mathcal{N}_\theta(x_i)\right) \right]
    \label{eq:steepest_gd}
\end{equation}

Where $\eta$ is the \textit{learning rate} that is given as an external parameter (\textit{hyperparameter}) to the model. 
Eq.\ref{eq:steepest_gd} implements the steepest gradient descent. However, in most machine learning optimizers a variant 
of this algorithm is computed. Namely, the gradients are calculated for each sub-set, often called \textit{mini-batch} of the 
whole training set. During the training, within each \textit{epoch} as many mini-batches as are needed to make the full 
dataset, are minimized in series. This approach, called \textit{stochastic} gradient descent (SGD), is less expensive 
in terms of memory and offers well established convergence properties that outperform classical steepest descent methods 
\cite{Zhao2021_sgd}. In fact, the ``noise'' affecting the updates induced by the minimization of a small sub-set can be beneficial 
for escaping saddle points which may trap the search. 
Over the years, different variants of SGD have been proposed. The ``momentum'' calculation \cite{Polyak1964}, 
that keeps track of the magnitude and direction of the updates and determines the next update as a linear combination 
of the gradient and the previous update, was first applied to SGD \cite{Backpro_1986}. Later, adaptive approaches 
have aimed at tuning the learning rate differently for each parameter (AdaGrad \cite{Adagrad}, ADAM \cite{ADAM}).
Later in the text, the DL models employed have adopted the ADAM optimizer. 
It is worth mentioning that, though the most widely used, gradient descent approaches are not the only strategies that 
have been explored to minimize the loss function. Evolutionary algorithms inspired by natural selection mechanisms and 
tensor optimization techniques developed in the quantum-many body field have been employed as well \cite{EA_1999, DMRG_Stoudenmire}.

\begin{figure}[H]
    \centering
    \includegraphics[width=\textwidth]{figures/Intro/sgd.pdf}
    \caption{\textbf{Gradient descent and Stochastic gradient descent. a)} Depicts the trajectory of the error score in 
     a 2D loss function in case of gradient descent with line search from three different initializations. The 
     algorithm converges to the global minimum when it starts from points 1 and 3 while it fails when it starts from 
     point 2, located outside the valley of the global minimum. \textbf{b)} The same experiment solved with stochastic 
     gradient descent achieves good convergence also when initialized from point 2. Indeed, the ``noise'' introduced 
     by the mini-batches drives the updates outside the wrong valley and allow the convergence to the global minimum. 
    Adapted from \cite{prince2023understanding}}
    \label{fig:sgd}
\end{figure}

\subsection{Backpropagation and automatic differentiation}

At this stage, a crucial question arises: \textbf{how can SGD be efficiently implemented in practice?}
ANNs often involve millions or even billions of parameters $\theta$, thus computing exact derivatives with respect to 
such a large number of variables, for every batch and across multiple epochs, requires highly optimized algorithms 
and hardware acceleration.
Indeed, the practical feasibility of training neural networks was significantly advanced by the seminal work of Rumelhart, 
Hinton, and Williams in 1986 \cite{Backpro_1986}, which introduced the efficient \textit{back-propagation} technique that made 
large-scale training computationally tractable.

Back-propagation bases its principles on the fundamental \textit{chain rule} proposed by Leibniz in 1676 to calculate 
derivatives of function compositions. In fact, a ANNs can be seen as a composition of functions (the layers) in which 
at each stage the layer $i$-th processes the output of the layer $(i-1)$-th. Formally, one can write: 

\begin{equation}
    \mathcal{N}_{\theta} = f_\theta^{(L)} \circ f_\theta^{(L-1)} \circ ... \circ f_\theta^{0}
    \label{eq:composition}
\end{equation}

Where $f_\theta^{(L)}$ is the function representing the $L$-th layer of neurons of parameters $\theta$. The gradient 
of the loss function with respect to the parameters is therefore expressed, according to the chain rule as: 

\begin{equation}
    \nabla_{\theta^{(l)}} L  = \frac{\partial{(L)}}{\partial{f^{(L)}}} \cdot \frac{\partial{f^{(L)}}}{\partial{f^{(L-1)}}} 
    \cdot ... \cdot \frac{\partial{f^{(l)}}}{\partial{\theta^{(l)}}}
    \label{eq:chain_rule}
\end{equation}

This method decomposes the calculation of the full gradient into a sequence of smaller, local derivatives, each 
associated with the intermediate states of the network. During the forward pass, the activations and local 
Jacobians are computed and stored in memory, a process sometimes referred to as \textit{forward accumulation}. 
These stored quantities are then systematically reused during the backward pass to propagate gradients from the 
output layer back through the network parameters. This reuse of intermediate computations is what makes the 
algorithm highly efficient, and it is from this backward flow of gradients that the name \textit{back-propagation} 
originates.  

In modern machine learning frameworks such as PyTorch \cite{paszke2019pytorch} and TensorFlow 
\cite{abadi2016tensorflow}, back-propagation is implemented through the construction of a 
\textit{computational graph}. Each node in the graph represents an operation, while edges encode the dependencies 
between operations. Once the forward pass has been executed, this graph allows the backward pass to be automatically 
derived and executed, efficiently parallelized across GPUs or TPUs. In fact, back-propagation is a specific instance 
of a more general family of techniques known as \textit{automatic differentiation} (or \textit{algorithmic 
differentiation}) (AD) \cite{Griewank2000EvaluatingD, baydin2018}, and more precisely corresponds to 
\textit{reverse-mode automatic differentiation}. This connection highlights that back-propagation is not 
unique to neural networks, but rather belongs to the rigorous and general framework for computing exact 
derivatives of functions defined by computer programs.  

% It is useful to contrast automatic differentiation with two other approaches for computing derivatives. 
% \textit{Symbolic differentiation}, as implemented for instance in computer algebra systems, manipulates expressions 
% analytically but suffers from expression swell and is not well suited for large computational graphs. 
% \textit{Numerical differentiation}, based on finite-difference approximations, is conceptually simple but prone to 
% numerical instability and scales poorly with the number of parameters. In contrast, automatic differentiation 
% combines the exactness of symbolic methods with the efficiency of numerical approaches, making it particularly 
% well suited for large-scale models such as neural networks.  

\begin{figure}[H]
    \centering
    \includegraphics[width=\textwidth]{figures/Intro/backpro.pdf}
    \caption{\textbf{Schematic of Back-propagation on computational graph. a)} The training inputs $x_i$ are processed 
    in the forward pass by the network that produces intermediate states $y_{1,2}$ using weights $w_i$, and the final 
    output $y_3$. \textbf{b)} The error $E$ computed between the output and the estimate $t$ is propagated backward 
    and the gradients with respect to the weights $\nabla_{w_i} E = (\frac{\partial{E}}{\partial{w_1}},..., \frac{\partial{E}}{\partial{w_6}})$
    are obtained from the chain rule of derivatives. Adapted from \cite{baydin2018}}
    \label{fig:sgd}
\end{figure}

\newpage

\section{Convolutional Neural Networks (CNNs)}

It was anticipated earlier in the Chapter that the network design can strongly impact the learning of the mapping 
$\mathcal{M}$. In the specific case in which $\mathcal{X}, \mathcal{Y}$ represent sets of natural images, neural 
networks with layers employing convolution operations have shown to outperform fully connected ones \cite{Fukushima1982,LeCun1989Handwritten, LeCun1989Backpropagation}. 
The reasons behind the success of convolutions are intuitively explained as follows: 

\begin{itemize}

    \item Images are inherently high-dimensional, as they are defined on 2D or 3D grids. As a result, the number of 
    parameters required to connect each pixel in a fully connected neural network grows rapidly and soon becomes 
    intractable. Convolutional layers address this issue by employing a \textit{kernel}: a small trainable filter 
    that is convolved with the input image or intermediate feature maps. This mechanism enables weight reuse across spatial 
    locations, drastically reducing the number of trainable parameters, a property commonly referred to as 
    \textit{parameter sharing}. Furthermore, the localized receptive field of the kernel induces a form of structured 
    sparsity, effectively regularizing the model by constraining the mapping to be approximated using compact, spatially 
    coherent information.

    \item  In images, spatially adjacent pixels are typically correlated, while uncorrelated variations are often 
    attributable to noise. Fully connected layers process each pixel independently, requiring the network to learn 
    spatial dependencies solely through training. Convolutional layers, in contrast, explicitly exploit local spatial 
    correlations by aggregating information from neighboring pixels, thereby embedding structural priors into the 
    architecture. (see Fig. \ref{fig:conv})

    \item The image of a tree and another image of the same tree, translated require a different tuning of the
    weights in a fully-connected layer, although the interpretation of the image has not physically changed. 
    Learning a different parameter configuration for each different translation of the same object is highly inefficient. 
    On the contrary, an additional benefit of the parameter sharing in convolutional layers is the equivariance to 
    translation that inherently optimize the learning. This property allows the output to respond to the input change 
    in the same way. Note that this equivariance is not valid for other geometrical transformations like rotation and 
    magnification.

\end{itemize} 

% motivation
We have seen in Chapter \ref{chap:bcdi} that BCDI data is in the form of 3D images in which peculiar spatial structures 
made of peaks and fringes are clearly visible, hence the natural choice of convolutional neural networks. Moreover, the 
tasks addressed in this PhD thesis, namely the gap-inpainting and the phase retrieval, are classified as inverse problems 
where the goal is to reconstruct missing or unmeasured information from incomplete observations. As highlighted in recent 
surveys and related works \cite{Review_CNN_2020, CNN_inverse2017}, CNNs are particularly well suited for these problems 
because they can learn powerful image priors from data, act as regularizers even without extensive training 
(as shown in the Deep Image Prior framework \cite{Ulyanov_2020}), and can be combined with physics-based models to 
enforce data consistency.

Let us present now the building blocks of a typical convolutional neural network.

\subsection{Convolutional layer}

It needs to be clarified here that to improve computational efficiency, machine learning libraries typically implement 
the \textit{cross-correlation} operation rather than a convolution. The difference is that the convolution flips the kernel before 
computing the sum, whereas cross-correlation does not. This simplifies the implementation with no expressive power loss, 
since the weights are learned during training but in turn breaks the commutative property of the convolution. 
Given an input vector $\mathbf{x}$ with $N$ entries the 
output of a convolutional layer $\mathbf{y}$ with $M$ entries can therefore be written as a matrix - vector multiplication of 
the type: 
\begin{equation}
    \mathbf{y} = \mathcal W \mathbf{x}
    \label{eq:conv_op} 
\end{equation}

Where $\mathcal W \in \mathbb{R}^{M \times N}$ is a Toeplitz matrix, where each descending diagonal contains a kernel entry, 
arranged such that the matrix performs the cross-correlation as a linear operation.

The initialization of a convolutional layer in typical machine learning libraries involve the definition of: 
\begin{itemize}
    \item An integer number of kernels. This number corresponds to number of filters through which the input is 
    processed in a convolution operation. Each filter is normally associated to a different feature of the input 
    image that is extracted by the corresponding kernel. The output of the convolutional layer will 
    have an extra dimension (called \textit{channel}) with size corresponding to the number of filters.  

    \item The integer-valued size of the kernel. In 2D convolutions the kernel is a matrix and similarly 
    extended to a 3D tensor for three-dimensional inputs. It controls the scale of the features that the convolutional 
    layer can ``see''. Small kernels highlight small features and vice-versa.

    \item The padding parameter which handles the convolution at the borders of the image, where the kernel would 
    otherwise extend beyond the available data. It determines whether the output feature map maintains the same spatial 
    dimensions as the input (``same'' padding) or is reduced in size (``valid'' padding).

    \item The stride parameter: This integer-valued parameter (matrix in 2D and tensor in 3D) indicates the lateral shift 
    that is applied to the kernel across the image. If greater than 1 some pixels/voxels are skipped resulting in a 
    sort of sub-sampling allowing to filter out some low-level features. 
    
    \item The dilation rate, which is another integer-valued parameter (matrix in 2D and tensor in 3D) that allows 
    to insert holes between consecutive elements of the kernel. In this way the actual size of the kernel is increased 
    by the dilation value, but the overall number of weights is unvaried. This enables to cover a larger area of the input 
    without increasing the number of trainable parameters.

\end{itemize}

Additional features of a typical convolutional layer include the addition of trainable biases and the initialization and 
regularization of the kernels and biases. Lastly, a non-linear activation is usually applied to the output of 
the convolutional layer. 

We mention here the existence of so-called \textit{transposed convolutions} which are sometimes used in CNNs. Here the 
conceptual difference is that Eq.\ref{eq:conv_op} is computed with the transposed Toeplitz matrix $\mathcal{W}^T$. 
Transposed convolutions naturally produce an output that is larger 
than the input, because they reverse the spatial reduction of a standard convolution. For this reason, they are often used 
to replace explicit \textit{up-sampling} layers, which increase the spatial size through fixed interpolation (e.g. nearest-
neighbor or bilinear). Unlike standard interpolation, transposed convolutions perform a \textit{learned} up-sampling, 
where the network optimizes the kernel weights to reconstruct high-resolution features in a data-driven way. 

\begin{figure}[H]
    \centering
    \includegraphics[width=\textwidth]{figures/Intro/conv.pdf}
    \caption{\textbf{Schematic of 2D convolutional layer:  a) } The convolution is operated over a region of the image 
    with same size as the kernel ($3 \times 3$). The result of the multiplication of each pixel value $(x_{12},...,x_{34})$ 
    times the weight $(w_{11},...,w_{33})$ is then summed into a scalar $h_{23}$. \textbf{b)} The same operation with 
    the same kernel is performed to a shifted window of the input image.
    Adapted from \cite{prince2023understanding}}
    \label{fig:conv}
\end{figure}

\subsection{Pooling Layer}

After the convolutional layer and the activation function, it is often the turn of a \textit{pooling} layer. For the 2D case, 
the output of the layer at specific location $i,j$ is the result of a filtering operation on the neighboring pixels 
of the input. For instance, the ``Max Pooling'' layer replaces a rectangular region of the input with its maximum value, 
thus retaining only the strongest activation within that area. In contrast, the ``Average Pooling'' layer 
assigns to the output pixel the mean value of all pixels in the corresponding region. 
The benefits deriving from the pooling layer are mostly twofold. First, the size of the output is reduced by a factor 
proportional to the area of the pooling window (or the volume in the 3D case), thus lowering the computational 
and memory cost of subsequent operations. Second, by condensing information into a smaller region, pooling encourages 
the network to learn representations that are more robust to small translations. For instance, with Max Pooling, 
a shift of the most activated pixel within the pooling window does not affect the output. This behavior can be 
interpreted as an implicit prior that biases the optimization towards translation-invariant approximations. 

\begin{figure}[H]
    \centering
    \includegraphics[width=\textwidth]{figures/Intro/conv_maxpool.pdf}
    \caption{\textbf{Example of Convolutional and Max pooling layers.} A simulated 2D BCDI pattern is input image 
    (first row) of the CNN model presented later in Chapter \ref{chap:phase_retrieval}, after the training. The output of the first 
    convolutional layer + LeakyReLU activation is displayed in the second row (first 5 channels) while the output of 
    the subsequent Max Pooling layer is displayed in the third row. Notice the similarity between the two rows despite 
    the halved size of Max Pooling output. This shows that it efficiently condenses the information into smaller sizes.
    % Moreover, one can notice how different channels extract different features, some of which (0 and 4 in this case) 
    % have larger amplitude, thus ``importance'' to the network.
    }
    \label{fig:convmaxpool}
\end{figure}


The typical convolutional block is therefore composed of these three layers (convolutional, activation, pooling). 
A sequential application of the convolutional blocks to a 2D or 3D input image is often called \textit{encoder} and 
can reduce the dimension of the 
feature map to as many 1D vectors as the number of filters of the last convolutional layer. During the training, this 
reduced representation of the input is more and more driven to capture the essential information in a lower-dimensional 
space. At this point, for \textit{classification} tasks the feature map is flattened into a single 1D vector 
being the input of a fully connected layer. The output of this last layer is then usually interpreted as the probability 
score for the input image to belong to a specific class. Among the milestone CNN models for classification, it is worth 
highlighting LeNet-5, introduced by LeCun in 1998 \cite{lecun1998gradient}, and AlexNet, which revolutionized the 
field by leveraging more efficient GPU-based training \cite{krizhevsky2012imagenet}.

In \textit{image generation} tasks, the lower-dimensional space, also called \textit{latent space} serves as input 
to a sequence of 
\textit{deconvolutional} blocks, where \textit{transposed} convolutions combined with activation functions 
progressively reconstruct the output image at the desired resolution. This sequence of deconvolutional layers 
is referred to as the \textit{decoder}, as it mirrors the encoder. In some architectures, up-sampling layers 
followed by standard convolutions are employed instead of transposed convolutions. Encoder-decoder architectures 
for image generation were first introduced by Hinton in 2006~\cite{hinton2006reducing}. This class of models is employed 
in a variety of tasks such as de-noising, image compression, inpainting, 
and segmentation. Although other types of models are also used for image generation, including Generative 
Adversarial Networks (GANs) \cite{goodfellow2014generativeadversarialnetworks} and Vision Transformers 
(ViTs) \cite{dosovitskiy2021imageworth16x16words}, in this manuscript we restrict our focus to 
encoder-decoder architectures.


\subsection{Skip connections}

Before concluding the chapter it is worth mentioning the concept of \textit{residual connections}, or more in 
general \textit{skip connections} which has been introduced in deep convolutional neural networks with the ResNet model 
\cite{he2015deepresiduallearningimage}. In this type of architecture the output feature maps in the encoder blocks
is summed to the input of the decoder blocks with corresponding size. In other cases, like the U-Net architecture 
\cite{ronneberger_u-net_2015} a concatenation of the two feature maps along the channel dimension is performed instead 
of the sum. The goal of the skip connections is to facilitate the training of deep generative networks by ``reminding'' 
the model about the features extracted in the beginning of the forward process. In fact, the more difficult training 
of deep networks as compared to shallow ones is attributable to the more unstable gradients \cite{balduzzi2018}. Tiny 
changes in the input result in a completely different gradient. A study conducted by Li \textit{et al.} has shown 
the benefits of skip connections through a study of the loss function landscape, which tends to have less saddle points 
and local minima \cite{li_visualizing_2017}. \\

This paragraph concludes the introductory background on neural networks. More broadly, the chapter complements the 
presentation of the physics and conventional data analysis of the BCDI technique from the first two chapters, by 
introducing the concepts, terminology, and insights needed to approach the content of the following three chapters.



% CHAPTER 4 - Deep Learning for Detector Gaps Inpainting
\chapter{Deep Learning for Detector Gaps Inpainting}
\label{chap:inpainting}

In this chapter the ``detectors' gaps problem'' in Bragg Coherent Diffraction Imaging and our approach to solve it
using Deep Learning are discussed. The main state-of-the-art measures are presented briefly and
the topic of image inpainting with Deep Learning is introduced. The focus will then shift to our works that led
eventually to the optimal ``Patching-based'' approach that can also be found in the published paper entitled
 \textit{``Patching-based deep learning model for the Inpainting of Bragg Coherent Diffraction patterns affected 
 by detectors' gaps''} (\url{https://doi.org/10.1107/S1600576724004163}). The chapter is closed with some analyses 
 of the performances of the DL models in a variety of simulated and experimental cases.  

\section{The ``Gap Problem''}\label{sec:gaps}

At time of writing, standard BCDI experiments employ pixelated photon counting detectors to acquire the diffraction
patterns. These detectors can guarantee high spatial resolution, noise-free counting and fast read-out times. Two examples 
of these devices, currently used at the ID01 beamline are the MAXIPIX and EIGER detectors \cite{ponchut_maxipix_2011, Eiger_Johnson_2014}.
These detectors are often built by tiling together several sensing chips in order to cover a larger area, and are
typically bonded to an Application-Specific Integrated Circuit (ASIC) using bump bonding. 
This implies the presence, in the overall sensing region, of vertical and/or horizontal stripes that are not sensitive
to the impinging radiation. The width of these lines varies depending on the device but normally does not exceed the equivalent 
of some tens of pixels. Specifically, for the MAXIPIX detector the gap size is 6 pixels wide while the EIGER has two types of larger gaps of
12 pixels and 38 pixels width.
The detector gaps problem does not affect BCDI only, but it is shared among other x-ray techniques that deal with single photon-counting
pixelated detectors and/or beamstops.
We have seen in chapter \ref{chp:intro} that during a BCDI scan the 2D images acquired by the detector are stacked to form
a 3D array. This leads these lines to become planes of missing signal in the dataset.
The problems arise when reconstructing the data affected by these gaps. In fact, these regions of non-physical zero intensity
deceive the Phase Retrieval algorithms inducing the presence of artifacts in the reconstructions\cite{carnis_towards_2019}.


\begin{figure}
    \includegraphics[width=\textwidth]{figures/Inpainting/gaps_intropdf.pdf}
    \caption{\textbf{Effect of detector gaps in BCDI reconstructions} 
    \textbf{(a)} The central xz slice of an experimental diffraction pattern. \textbf{(b)} The same slice of the diffracted
    intensity calculated from the retrieved object. \textbf{(c - d)} xz slice of the modulus and phase respectively of the particle
    obtained from the phasing of the gap-less dataset. \textbf{(e)} Same slice as in \textbf{(a)} with an artificially added
    6 pixel-wide, cross-shaped gaps to mimic the detector's ones. \textbf{(f)} The same slice of the diffracted
    intensity calculated from the retrieved object when not masking the gap regions. \textbf{(h - g)} xz slice of the modulus and phase respectively of the particle
    obtained from the phasing of the gap-affected dataset. The distortions caused by the gaps are evident. }
    \label{fig:gap_intro}
    \end{figure}


It follows that the reliability of the reconstructions in this case is 
compromised as the strain distribution can be deeply affected by the artifacts. A good practice during standard BCDI experiments
is to avoid the gaps by moving the detector if possible. However, this tends to be problematic for the case of high-resolution BCDI, 
i.e. when the diffraction pattern measurement extends to higher q-values, thus covering more than one sensing 
chip and necessarily crossing a gap region. Under these circumstances it becomes important to reduce the amount of
artifacts deriving from the gaps. 


\section{State of the art}\label{sec:InpStateArt}

Here we will discuss the current strategies employed to treat the detector gaps. As someone could argue, the simplest
yet not practical, solution would be to slightly move the detector sideways and acquire a second full scan with the
gap hiding a different region of the same Bragg peak, and then merge the two measurements into a single gap-less one. 
This would in turn increase the acquisition time of more than 2X making it de facto never an option during standard experiments. 
Iterative phasing algorithms like PyNX allow the user to define a mask of the gap regions and ignore those pixels during the execution. 
In this way the quality of the reconstruction improves, but one can still notice the presence of high-frequency oscillations 
appearing in both object's modulus and phase. The origin of these artifacts can be found in the diffracted intensity calculated 
from the reconstructed particle as one can clearly see that the gap-regions is filled with nonphysically high intensity.

Another, more invasive, option is to \textit{fill} these gaps with an estimate of the intensity distribution that
would be there, before the phase retrieval. These tasks of filling gap in images is usually referred to as ``inpainting'' 
and it has been widely studied in the field of photography and imaging \cite{Elharrouss_2019,reviewInpainting2021}.
More recently Deep Learning models have taken the place of more traditional methods as they can attain higher accuracy
for more complex inpainting tasks \cite{reviewInpaintingDL2023}. 
For the case of scientific images 


\section{Model design}\label{sec:model}
\section{Patching approach}\label{sec:patching}
\section{Results in detector space}\label{sec:res_rec}
\section{Results in real space}\label{sec:res_real}
\section{Fine-tuning}\label{sec:finetuning}
\section{Performances assessment}\label{sec:performances}


% CHAPTER 5 - Deep Learning for Phase Retrieval
\chapter{Deep Learning for Phase Retrieval}
\label{chap:phase_retrieval}
\chapter{Phase Retrieval}\label{chp:phasing}


% CHAPTER 6 - Deep Learning for Phase Retrieval
\chapter{Automatic Differentiation for BCDI Phase Retrieval}
\label{chap:AD_phase_retrieval}
% In this last chapter an approach to the BCDI Phase Retrieval based on Automatic Differentiation will be discussed.
% It started from the necessity to 
% Unlke the DL model discussed above this method is iterative and

In this chapter a different approach to the BCDI phase retrieval will be presented. It originated from the need to resolve 
those cases in which neither standard alternating algorithms, nor the DL assisted PR can succeed to converge to a satisfactory 
reconstruction. The developed approach differs from the alternating projections algorithms classically used for 
the Fourier PR, as it is formulated as minimization problem solved with gradient descent (GD). The gradients however are computed 
through the efficient automatic differentiation (AD) enabled by graph-based differentiable programming packages like Tensorflow and 
PyTorch, accelerated on GPU. For this reason one could see the AD approach as unsupervised machine learning on a single training 
dataset.\\ 
The GD - based optimization is fundamentally different from fixed point alternating projections. Here one could qualitatively say 
that if the latter switches between real and reciprocal space applying constraints in both, the former initializes a 
complex object and updates at each cycle its modulus and phase using the gradients, with respect to them, of the differences 
between the observed and calculated diffracted intensities. In this way, the knowledge on the particle can be implemented 
by initializing the object with some physical constraints or adding regularization terms that will drive the updates 
towards more reasonable solutions. \\

After mentioning the most relevant literature on AD, and more generally GD-based, phase retrieval for CDI, 
we will present our formulation of the problem and the results obtained on simulated and experimental BCDI patterns. 

\section{State of the Art}
\section{Model implementation}
In an AD-driven optimization problem some trainable parameters are initialized. In the first basic formulation these 
trainable parameters can be the values of the voxels corresponding to the modulus $m$ and the phase $\varphi$ of the complex objects 
that represents the solution of the PR problem. All of these voxels contribute to the creation of a simulated 
diffracted intensity pattern via the forward model $I_{calc} = |\mathcal{F}\left\{ me^{i\varphi} \right\}|^2$. Subsequently, 
the gradients of a metric (loss function) that estimates the distance between the observed BCDI pattern $I_{obs}$ and $I_{calc}$ are calculated 
with respect to each of the trainable variables with automatic differentiation. At this point the value of each of these voxels is 
updated using a chosen optimizer (SGD, ADAM, etc.) and a given learning rate. The Tensorflow library allows for an easy 
implementation of the trainable variables and loss function and handles gradient operations with predefined methods. It is therefore 
straightforward to run the optimization as it follows the same structure of a deep learning model, with less trainable parameters and 
for a single data. 

However, such simple formulation of the complex object as mere real-valued variables is not optimal for a non-linear and non-convex 
inverse problem such the Fourier phase retrieval. In fact, many non-physical modulus-phase configuration could yield a 
$I_{calc}$ that is close to  $I_{obs}$. The presence of these local minima is the reason why, in conventional PR, algorithms 
like hybrid input-output, capable of escaping them, are employed. 
For this reason, the formulation of the complex object to be optimized has embedded some physical considerations that 
helped to restrict the solution space, removing some of the many local minima of populating the landscape of the BCDI PR. 
Moreover, it was shown by Marchesini in \cite{marchesini_unified_2007}
that steepest GD and the more sophisticated conjugate GD are more prone to get stuck in local 
minima, reason why they are not commonly utilized for Fourier PR. However, the active research field of machine learning has 
brought important advancements in the formulation of efficient and robust optimizers based on stochastic gradient descent with 
powerful features like Nesterov or adaptive momentum (ADAM \cite{ADAM}). These GD techniques are more robust to local minima, 
since the gradient is computed on mini-batches of trainable variables rather all of them (stochastic rather than classical steepest GD), 
and converge faster thanks to the ``memory'' of previous steps. Additionally, they are often wrapped into handy classes, ready to use, 
in Tensorflow and Pytorch libraries.  

\subsection{Object's shape}

The formulation of the object's shape has started considering the typical crystalline samples that are studied with the BCDI technique
and the requirements the modulus of the reconstructed object need to fulfill to be considered a ``good solution''. 
Usually, successful reconstruction show a \textit{homogeneous} modulus, sometimes quantitatively assessed through the 
mean-to-max metric \cite{Frisch2023CuAgCatalysts} , \cite{Grimes2024CatalystStrain}, as in standard BCDI the form factor is 
approximated uniform across all the scattering sites. Enforcing a homogeneous modulus by construction limits the search space 
and helps the convergence to the solution. 
It follows that parametrizing the \textit{surface} of the support, and setting to 1 the inside, is much more advantageous than optimizing 
the full 3D volume. This approach, already proposed by Scheinker and Pokharel in \cite{scheinker_adaptive_2020}, 
also significantly reduces the number of variables to optimize.

An additional consideration is that the probed samples are crystalline, thus often \textit{faceted} and \textit{convex}. 
Therefore, one could simplify even more the construction of the object shape by building a certain amount of planes in the 
3D space and obtain the support from the volume that lies inside the intersections of all them. This would remove the possibility 
to have spikes or rough surfaces that might satisfy some local minimum but wouldn't represent a crystal. Moreover, with this 
representation the number of trainable variables would be further reduced. 

According to this scheme the relevant parameters to be optimized are the angles $\theta$ and $ \varphi$ of the spherical 
coordinates and the length $d$ of a given number $N$ of the so-called \textit{half-spaces}. 
More formally, the normals $n_i$ for each of the $N$ half-spaces are defined with a pair of ($\theta , \varphi$) that  its
orientation in space (Eq. \ref{eq:normal_vector}). Subsequently, only the intersection of those $(x,y,z)$ coordinates for which the dot product with 
each $n_i$ is smaller than the length $d_i$ is considered as support (Eq. \ref{eq:convex_hull}).


\begin{equation}
    \mathbf n_i \;=\;
    \begin{pmatrix}
    \sin\varphi_i\cos\theta_i \\[6pt]
    \sin\varphi_i\sin\theta_i \\[6pt]
    \cos\varphi_i
    \end{pmatrix},
    \label{eq:normal_vector}
    \end{equation}
    
    \begin{equation}
    \mathcal S \;=\;
    \bigcap_{i=1}^N
    \Bigl\{\mathbf x=(x,y,z)\in\mathbb R^3 : 
    \mathbf n_i\cdot\mathbf x \le d_i\Bigr\},
    \label{eq:convex_hull}
    \end{equation}

A schematic representation of this construction is provided by Fig. \ref{fig:support_construction}.
 
With this approach the user needs to provide a number of half-spaces as hyperpareter meaning that a sort of prior knowledge 
on the sample can be leveraged in these regards as well. However, this number doesn't have to be precisely the number of 
facets expected. In fact, a large $N$ is often advised for unknown sample shape such that even roundish objects can be 
retrieved. In case of well faceted samples the large $N$ is a minor problem as many $n_i$ will be automatically aligned to the 
same $(\theta_i, \varphi_i, d_i)$ at the cost of some more trainable parameters. 

The first drawback of this convex-hull parametrization is that concave objects can't be retrieved. However, these cases 
are much less frequent in typical BCDI experiment. The second limitation is that this formulation is incapable of modeling 
defects that would zero the contribution of the object's modulus to the diffraction pattern \cite{favre-nicolin_analysis_2010}. 
A correct BCDI reconstruction of particles affected by this type of defects presents ``holes'' inside the hull in correspondence of the 
defect. However, the current model cannot address this type of features as the support is by construction fully homogeneous 
inside the borders. Further developments of the algorithm could indeed aim at a more complete formulation of the construction 
of the object modulus. 

\begin{figure}[H]
    \centering
    \includegraphics[width=.8\textwidth]{figures/AD/AD.pdf}
    \caption{Construction of the convex hull with half-spaces expressed with spherical coordinates }
    \label{fig:support_construction}
\end{figure}

The last important consideration of this parametrization is that the support $\mathcal S$ is sharply divided into a binary 
variable (1 inside and 0 outside) thus leading to differentiability problems. In fact, in such a way the gradients, essential for the 
support update, are not defined. For this reason $\mathcal S$ is first passed through a sigmoid function controlled by a
hyperparameter $\epsilon$ responsible for the smoothening of the support borders. This measure can also be seen as a control of the 
resolution of the object. Additionally, a mildly steep sigmoid in the early stage of the optimization can function help retrieving 
a low resolution estimate of the support, that can be further refined adjusting the $\epsilon$ parameter. \\

\subsection{Object's phase}
The parametrization of the object's phase is more challenging. From a qualitative point of view, the prior knowledge 
that can be exploited for a tailored implementation, is limited to the awareness that a physically meaningful atomic displacement 
field cannot have ``too many'' sharp variations. This observation is translated into code by smooth functions parametrization or total 
variation (TV) regularization of the object's phase. While the former would enforce smoothness by construction the latter 
would operate adding a penalty to the data-fidelity term of the loss function for non-smooth solutions. Both approaches have been 
explored and are here reported. \\

Forcing a scalar field defined on an  $L\times H\times W$ grid to exhibit smooth behavior is equivalent to seeking a 
sparse representation of that field—that is, to concentrating its essential information into far fewer degrees of freedom 
than the original $L\times H\times W$ samples. 
Concretely, one looks for a change of basis in which the field can be written as a linear combination of a hierarchy 
of modes or atoms, ordered by “importance.” In a Fourier or wavelet expansion, for instance, the expansion coefficients 
are naturally sorted from largest (low‑frequency or coarse‑scale modes) to smallest (high‑frequency or fine‑scale modes). 
Retaining only the largest coefficients both compresses the data and removes rapid oscillations, yielding an inherently 
smoother reconstruction. Equivalently, in the matrix case a Singular Value Decomposition (SVD) identifies an orthonormal 
basis in which only a few singular values are nonzero; by truncating to the top singular values one obtains a low‑rank 
— and thus smoother—approximation \cite{golub1996matrix}. For higher dimensional data, this same principle underlies 
higher-order generalizations of the SVD—Tucker/HOSVD, CP, Tensor-Train, and T-SVD—each of which orders multilinear 
“modes” by their singular-value (or eigenvalue) strength, and truncating to a small subset produces both compression 
and smoothness \cite{Kolda_TT}. 

In this case the Tucker decomposition was chosen, among the several possible methods, for its simplicity of implementation 
with the Tensorflow library and for the suitability for moderately low dimensions \cite{Oseledets_TT}. 
For a 3D tensor the Tucker decomposition is done as follows: 

Considering \(\mathcal{\varphi} \in \mathbb{R}^{L \times H \times W}\) the 3D object's phase. The Tucker decomposition expresses \(\mathcal{\varphi}\) as:
\[
\mathcal{\varphi} = \mathcal{G} \times_1 U^{(1)} \times_2 U^{(2)} \times_3 U^{(3)},
\]
where:
\begin{itemize}
  \item \(\mathcal{G} \in \mathbb{R}^{R_1 \times R_2 \times R_3}\) is the \textbf{core tensor},
  \item \(U^{(1)} \in \mathbb{R}^{I \times R_1}\), \(U^{(2)} \in \mathbb{R}^{J \times R_2}\), and \(U^{(3)} \in \mathbb{R}^{K \times R_3}\) are the \textbf{factor matrices},
  \item \(\times_n\) denotes the mode-\(n\) tensor-matrix product.
\end{itemize}

In index notation, this becomes:

\[
\mathcal{\varphi}_{i,j,k} = \sum_{\alpha=1}^{R_1} \sum_{\beta=1}^{R_2} \sum_{\gamma=1}^{R_3}
\mathcal{G}_{\alpha,\beta,\gamma} \cdot U^{(1)}_{i,\alpha} \cdot U^{(2)}_{j,\beta} \cdot U^{(3)}_{k,\gamma}.
\]

With this formulation the parameters $R_1, R_2, R_3$ are set by the user and define the ``storage space'' in which the 
information required to represent $\varphi$ has to be condensed. It is proven that for $R_i = L,H,W$ respectively, the 
tensor $\varphi$ is exactly represented. However, being the goal a spare representation of the object's phase these numbers 
are chosen significantly smaller than any of the sizes of the array. The Tensorflow implementation of the Tucker decompostion 
is rather straightforward as the function \texttt{tf.einsum()} takes care of the tensor contraction.


\subsection{Loss function}
Another important aspect of the model is the loss function. Typically, for inverse problems there is \textit{data fidelity}
term that in this case measures the distance between $I_{obs}$ and $ I_{calc}$ according to some metric, and other additional 
\textit{regularization} terms that guide the optimization process with physical constraints. 

%  data fidelity 
%  talk about batches 

\section{Results}
\subsection{Low-strain case}
\subsection{High-strain case}
\section{Conclusions}


% CHAPTER 7 - Conclusions
\chapter{Conclusions}
\label{chap:conclusions}
In this PhD dissertation the use of deep convolutional neural networks and algorithmic differentiation has been 
explored for the processing of BCDI data. Chapter \ref{chap:bcdi} has presented the physics of coherent x-ray diffraction 
on single crystal, highlighting the effect of internal lattice displacement on the data. Moreover, a short practical 
overview of the typical BCDI experiments was given as well. Chapter \ref{chap:phase_problem} was dedicated to the fascinating 
Fourier phase problem. The uniqueness conditions, the PR iterative algorithms and some insights on the 
high-strain case were discussed. In particular, the link between the effects of the high-strain on the diffraction pattern, illustrated 
in Chapter \ref{chap:bcdi} and the relative increased difficulty of the PR shown in Chapter \ref{chap:phase_problem} has 
been emphasized. What emerged in the discussion is that improved performance of the PR is obtained when the problem is 
\textit{regularized} with some prior knowledge that constrains or guides the search of the solution. Here, the connection 
to neural networks introduced in Chapter \ref{chap:dl_theory} becomes apparent. The strong inductive prior of convolutional 
neural networks for structured images, combined with a targeted data-driven strategy, was explored in Chapters \ref{chap:inpainting}
-\ref{chap:phase_retrieval}, yielding satisfactory results on two different kinds of inverse problems. 


Specifically, 
in Chapter \ref{chap:inpainting} the preliminary 
investigations and the development of a novel patching-based model for addressing detector gaps in BCDI data were presented. 
The results obtained on new simulated and experimental datasets confirmed the capability of convolutional neural networks 
to extrapolate information from structured images and to accurately predict the continuation of patterns within missing 
data regions. 
In Chapter \ref{chap:phase_retrieval} the goal of assisting standard iterative algorithms during the PR 
of highly-strained BCDI patterns has been achieved with the use of a convolutional neural network trained with the novel 
WCA loss function for the prediction of the \textit{reciprocal space phase}, unlike what present in the literature. 
What presented in Chapter \ref{chap:AD_phase_retrieval} instead moves away from the data-driven approach. Here, the
computational framework for automatic differentiation is leveraged for a physics constrained PR solved 
with gradient descent. The prior knowledge of uniform electron density inside the object support and the faceted nature 
of crystals' surfaces does not come from the data, nor from penalty terms but from construction constraints. 

For each result chapter, I will provide an outlook on potential directions for future developments in the field.

\begin{itemize}
    \item \textbf{DL-based Gap Inpainting.} The patching approach has proven to have many benefits, including faster training, 
    larger training datasets and possibility of targeted fine-tuning. The bottleneck of this method is however given by 
    the size of the gaps. Detectors like EIGER, have large gaps (12 - 38 pixels) for which the patching approach is not 
    suitable. One could get around this limitation increasing the patch size, however, in those cases other methods 
    leveraging some type of regularization during the reconstruction \cite{Chushkin2025} can offer a better alternative 
    to a DL approach. 
    Another interesting development could aim to reduce the number of repeated applications of the DL model along the gap. 
    In fact, given the typical structure of BCDI patterns and experimental conditions, the gap only affects a small region 
    of signal, often on long truncation rods streaks. An adaptive algorithm for the inpainting of those regions only, could 
    save time and computations discarding the inpainting of dark regions.
    More in general, it would be interesting to test the patching-based inpainting on data from other imaging techniques.   
    
    \item \textbf{DL-based Phase Retrieval.} The RSP prediction on patches is at the same time appealing and challenging. 
    The results obtained on independent patches are promising, but the stitching of the patches still represents a 
    challenge. As anticipated in the conclusion Chapter \ref{chap:phase_retrieval}, the design of a Recurrent convolutional 
    neural network could better address the problem. In particular, I would suggest an approach similar to what proposed 
    by Pinheiro and Collobert in \cite{Pinheiro2014RecurrentCN} could be considered. 
    There, instead of stacking many different layers to increase receptive fields, they used \textit{weight-sharing} recurrence, 
    i.e. the same convolutional layer is applied multiple times to its own output. This allows the network to iteratively 
    refine predictions and integrate increasingly larger context without adding new parameters.
    
    This approach could also break the symmetry problem that was evidenced in the end of Chapter \ref{chap:phase_retrieval}
    (see Fig. \ref{fig:non_centrosymm_LS}) 
    for which the predicted RSP always tend to show a radial symmetry like the average, over the whole training dataset, 
    of the diffraction intensity that the model receives as inputs. 

    \item \textbf{AD-based Phase Retrieval.} Concerning this project, in my opinion the potential is high and several 
    developments, additions, integrations can be foreseen. The high flexibility and efficiency of gradient-based 
    optimizations provided by modern machine learning libraries allows for relatively easy implementation of tailored 
    models. The first important extension to the current formulation should include the modeling of non-convex surfaces 
    as well. An idea that would maintain the half-spaces approach would be to define two or more convex volumes built 
    with the half-spaces method and find the final support with union, subtraction and intersection operations. 
    Additionally, other upgrades to the current model could make use of the observed intensity projection 
    (the projection on the modulus constraint set presented in Chapter \ref{chap:phase_problem}), combined with the 
    gradient descent, as well as the use of mini-batches to enable stochastic gradient descent, 
    for faster and more robust convergence. 
    
    More in general, the utilization of GPU-accelerated AD is confident to become a pivotal tool for efficient gradient-
    based optimization in various scientific disciplines characterized by intensive computational demands. This approach 
    is anticipated to gain prominence in the coming years, potentially surpassing data-driven methodologies in certain 
    applications \cite{baydin2018,STIERLE2024120380, 10487099}.

\end{itemize}



% Appendices
\appendix
\chapter{Additional Data and Methods}
\label{chap:appendix}
\chapter{Appendix}\label{chp:appendix}



% Bibliography
% \bibliography{biblio_06_23}
\printbibliography
\addcontentsline{toc}{chapter}{Bibliography}

% Annexes
\begin{appendices}%\renewcommand{\thesection}{\Alph{section}}
\appendixheaderon
%\section{Évolution des métriques au cours de l'entraînement}
\label{part:annexe}
\chapter{Appendix}\label{chp:appendix}

\end{appendices}

\end{document}
