In this PhD dissertation the use of deep convolutional neural networks and algorithmic differentiation has been 
explored for the processing of BCDI data. Chapter \ref{chap:bcdi} has presented the physics of coherent x-ray diffraction 
on single crystal, highlighting the effect of internal lattice displacement on the data. Moreover, a short practical 
overview of the typical BCDI experiments was given as well. Chapter \ref{chap:phase_problem} was dedicated to the fascinating 
Fourier phase problem. The uniqueness conditions, the PR iterative algorithms and some insights on the 
high-strain case were discussed. In particular, the link between the effects of the high-strain on the diffraction pattern, illustrated 
in Chapter \ref{chap:bcdi} and the relative increased difficulty of the PR shown in Chapter \ref{chap:phase_problem} has 
been emphasized. What emerged in the discussion is that improved performance of the PR is obtained when the problem is 
\textit{regularized} with some prior knowledge that constrains or guides the search of the solution. Here, the connection 
to neural networks introduced in Chapter \ref{chap:dl_theory} becomes apparent. The strong inductive prior of convolutional 
neural networks for structured images, combined with a targeted data-driven strategy, was explored in Chapters \ref{chap:inpainting}
-\ref{chap:phase_retrieval}, yielding satisfactory results on two different kinds of inverse problems. 


Specifically, 
in Chapter \ref{chap:inpainting} the preliminary 
investigations and the development of a novel patching-based model for addressing detector gaps in BCDI data were presented. 
The results obtained on new simulated and experimental datasets confirmed the capability of convolutional neural networks 
to extrapolate information from structured images and to accurately predict the continuation of patterns within missing 
data regions. 
In Chapter \ref{chap:phase_retrieval} the goal of assisting standard iterative algorithms during the PR 
of highly-strained BCDI patterns has been achieved with the use of a convolutional neural network trained with the novel 
WCA loss function for the prediction of the \textit{reciprocal space phase}, unlike what is present in the literature. 
The successful results attained by the DL + PyNX refinement method can significantly improve the BCDI technique by 
drastically reducing the computational time needed for the reconstructions of experimental data. Over the long term, 
the computational resources required to train the DL model are expected to be minimal relative to the substantial 
time and energy savings achieved through DL-based initialization of the PR.

Chapter \ref{chap:AD_phase_retrieval} instead moves away from the data-driven approach. Here, the
computational framework for automatic differentiation is leveraged for a physics constrained PR solved 
with gradient descent. The prior knowledge of uniform electron density inside the object support and the faceted nature 
of crystals' surfaces does not come from the data, nor from penalty terms but from construction constraints. 

Integrating the presented algorithms into the standard BCDI data analysis pipeline should be foreseen
to enable more systematic usage. So far, all three main models have been employed by ID01 users for inpainting and 
phase retrieval tasks, with experimental results soon to be published. However, before this integration some further 
developments could be considered.
I will provide an outlook on potential directions with the following.

\begin{itemize}
    \item \textbf{DL-based Gap Inpainting.} The patching approach has proven to have many benefits, including faster training, 
    larger training datasets and possibility of targeted fine-tuning. The bottleneck of this method is however given by 
    the size of the gaps. Detectors like EIGER, have large gaps (12 - 38 pixels) for which the patching approach is not 
    suitable. One could get around this limitation increasing the patch size, however, in those cases other methods 
    leveraging some type of regularization during the reconstruction \cite{Chushkin2025} can offer a better alternative 
    to a DL approach. 
    Another interesting development could aim to reduce the number of repeated applications of the DL model along the gap. 
    In fact, given the typical structure of BCDI patterns and experimental conditions, the gap only affects a small region 
    of signal, often on long truncation rods streaks. An adaptive algorithm for the inpainting of those regions only, could 
    save time and computations discarding the inpainting of dark regions.
    More in general, it would be interesting to test the patching-based inpainting on data from other imaging techniques.   
    
    \item \textbf{DL-based Phase Retrieval.} The RSP prediction on patches is at the same time appealing and challenging. 
    The results obtained on independent patches are promising, but the stitching of the patches still represents a 
    challenge. As anticipated in the conclusion Chapter \ref{chap:phase_retrieval}, the design of a Recurrent convolutional 
    neural network could better address the problem. In particular, I would suggest an approach similar to what proposed 
    by Pinheiro and Collobert in \cite{Pinheiro2014RecurrentCN} could be considered. 
    There, instead of stacking many different layers to increase receptive fields, they used \textit{weight-sharing} recurrence, 
    i.e. the same convolutional layer is applied multiple times to its own output. This allows the network to iteratively 
    refine predictions and integrate increasingly larger context without adding new parameters.
    
    This approach could also break the symmetry problem that was evidenced in the end of Chapter \ref{chap:phase_retrieval}
    (see Fig. \ref{fig:non_centrosymm_LS}) 
    for which the predicted RSP always tend to show a radial symmetry like the average, over the whole training dataset, 
    of the diffraction intensity that the model receives as inputs. 

    \item \textbf{AD-based Phase Retrieval.} Concerning this project, in my opinion the potential is high and several 
    developments, additions, integrations can be foreseen. The high flexibility and efficiency of gradient-based 
    optimizations provided by modern machine learning libraries allows for relatively easy implementation of tailored 
    models. The first important extension to the current formulation should include the modeling of non-convex surfaces 
    as well. An idea that would maintain the half-spaces approach would be to define two or more convex volumes built 
    with the half-spaces method and find the final support with union, subtraction and intersection operations. 
    Additionally, other upgrades to the current model could make use of the observed intensity projection 
    (the projection on the modulus constraint set presented in Chapter \ref{chap:phase_problem}), combined with the 
    gradient descent, as well as the use of mini-batches to enable stochastic gradient descent, 
    for faster and more robust convergence. 
    
    More in general, it is clear that the utilization of GPU-accelerated AD will become a pivotal tool for efficient gradient-
    based optimization in various scientific disciplines characterized by intensive computational demands. This approach 
    is anticipated to gain prominence in the coming years, potentially surpassing data-driven methodologies in certain 
    applications \cite{baydin2018,STIERLE2024120380, 10487099}.

\end{itemize}
