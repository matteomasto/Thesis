\section{Introduction}\label{chp:intro}

In this manuscript, the use of Deep Learning methods, and more in general of GPU accelerated optimizations, for the advance of 
the data analysis in Bragg Coherent Diffraction Imaging (BCDI) will be presented. However, before delving into the 
study’s developments, I would like to share with the reader a reflection that has taken shape over the course of this 
PhD, serving as a kind of preface. In particular, I have come to observe that, unlike other more fundamental 
scientific investigations, this work originates from the practical limitations of the technique in question. 
It is indeed because the detectors are imperfect—unable to record flawless images due to gaps, or incapable of 
capturing phase information because its oscillations are too rapid—that one is compelled to manipulate the available 
data with sophisticated algorithms. And, as often happens in science, compensating for these technical shortcomings 
leads to the development of tools rooted in the most abstract realms of mathematics and information theory. How much 
missing information can one extract from a signal? How can it be extracted, and under what conditions? In which 
circumstances is it easier, and why? Thus, a fascinating world opens up not when we directly investigate the foundations 
of matter, but when we examine \textit{how} we go about investigating them. 

Although this manuscript is ultimately focused on the specific cases of BCDI gap inpainting and phase retrieval, 
I hope to convey at least a bit of the wonder and awe that comes from knowing that such applications draw their roots 
from far deeper, more general, complex, and abstract themes.

\subsection{PhD objectives}

As we will soon present in details, we can anticipate that BCDI is a powerful imaging-microscopy technique performed at synchrotron 
and XFEL facilities and for its non-invasive, high spatial resolution, investigation of the atomic structure of single crystal 
nano-particles it has already been proved successful in many diverse fields. In fact, the study of the internal 
strain distribution, defects population and morphology at the nanometer scale (typical BCDI resolution is of the order of 10 nm) 
under various physio-chemical environments is of crucial importance for fundamental science as well as for engineering 
applications \cite{BCDI_review2024}. Since the very first use in 2001 by Prof. Ian Robinson and coauthors \cite{Robinson_gold_2001} for the imaging 
of gold particles, BCDI has been employed for the analysis of materials of relevance for nanotechnology and electronics 
\cite{Favre-Nicolin_2010} as well as for Li-ion and Na-ion batteries \cite{Singer2018, Serban2024}, catalysis 
\cite{atlan_imaging_2023} and recently on biological materials too \cite{Grunewald:ro5042}. 

However, this technique strongly relies on computer algorithms for the transformation of the acquired diffraction pattern into 
the real space particle shape and strain field. For this reason, in the past years, numerous efforts have been made  
to make the computational aspect of BCDI robust and reliable. Some of these developments will be presented later on, while 
here we want to highlight that the research in this field has become even more active and prosperous with the advent of machine 
learning (ML). Moreover, the recent upgrade to fourth-generation x-ray light source of many synchrotrons across the planet 
(MAX-IV - Sweden in 2017, ESRF-EBS - France and Sirius - Brazil in 2020 and others scheduled for the coming years) is 
boosting the crystalline nano-imaging techniques such as Bragg CDI and Bragg ptychography \cite{Li2022, Leake:il5024}. 
Hence, the need for fast and optimized ways to handle the large amounts of experimental data. It is exactly in this context 
that this study is framed, with the initial goal of exploring the advantages that machine learning algorithms can bring 
to the BCDI technique. \\
Specifically, the work has focused on two main tasks of primary importance for the reliability of the processed data. 
As briefly mentioned above and discussed more in details later, the detectors employed for x-ray imaging techniques 
have the main limitation of not being capable of record the phase of the impinging x-ray light, thus leaving its retrieval 
to computer algorithms. Moreover, these detectors are built with some non sensing 


