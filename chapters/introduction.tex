\section{Preface}\label{chp:intro}

In this manuscript, the use of Deep Learning methods, and more generally of GPU-accelerated optimizations, for the advance of 
data analysis in Bragg Coherent Diffraction Imaging (BCDI) will be presented. However, before delving into the 
study's developments, I would like to share with the reader a reflection that has taken shape over the course of this 
PhD, serving as a kind of preface. In particular, I have come to observe that, unlike other more fundamental 
scientific investigations, this work originates from the practical limitations of the technique in question. 
It is indeed because the detectors are unable to record flawless images due to gaps, or incapable of 
capturing phase information because its oscillations are too rapid—that one is compelled to manipulate the available 
data with sophisticated algorithms. And, as often happens in science, compensating for these technical shortcomings 
leads to the development of tools rooted in the most abstract realms of mathematics and information theory. This brings 
us to the field of inverse problems, i.e., the study of algorithms and numerical methods for handling incomplete data. 
How much information can one extract from a signal? How can it be extracted, and under what conditions? In which 
circumstances is it easier, and why? Thus, a fascinating world opens up not when we directly investigate the foundations 
of matter, but when we examine \textit{how} we go about investigating them. 

Although this manuscript is ultimately focused on the specific cases of BCDI gap inpainting and phase retrieval, 
I hope to convey at least some of the wonder and awe that comes from knowing that such applications draw their roots 
from far deeper, more general, complex, and abstract themes.

\section{Context, PhD objectives and manuscript outline}

Over the past two decades the field of Coherent X-ray Diffraction Imaging (CXDI or CDI) has combined the study of microscopic structures 
enabled by X-rays with the imaging world. This powerful connection has been enabled by the availability of high-brilliance and 
coherent X-ray sources and the developments of computer algorithms for Phase Retrieval (PR) \cite{fienup_reconstruction_1978, fienup_phase_1982,Luke_2004}. 
In fact, unlike conventional imaging methods, 
instead of relying on high-quality optics to form an image, CDI records only the intensity of the diffracted waves on 
a detector and then reconstructs the object by iterative algorithms. 
This approach removes the limitations imposed by imperfect lenses, enabling spatial resolutions set primarily by the 
wavelength and the numerical aperture of the scattering geometry. 
Since the first work by Miao \cite{Miao1998} the CDI technique has been widely employed in materials science for 
characterization at the nanometric scale of different functional materials. \cite{Neutze2000, Chapman2005, Schroer2008, Rodriguez2015}

When performed in Bragg geometry (Bragg CDI or BCDI), the technique enables precise 3D visualization and investigation 
of the internal atomic structure of single-crystal nanoparticles. Since its first demonstration in 2001 by Robinson 
et al. \cite{Robinson_gold_2001} for imaging gold nanoparticles, BCDI has proven to be a powerful tool for studying 
strain distributions \cite{pfeifer2006three, Robinson2009, Newton2010}, defect populations \cite{Favre-Nicolin_2010, Labat2015, Dupraz2017}, 
and particle morphologies under different physico-chemical conditions \cite{Carnis2021, FacetStrain2022, Chatelier2024, 
Grimes2024}. Over the years, it has been applied to a broad range of technologically relevant systems, including 
nanotechnology and electronics \cite{Favre-Nicolin_2010}, Li-ion and Na-ion batteries \cite{Singer2018, Serban2024}, 
catalysis \cite{atlan_imaging_2023}, and, more recently, biological materials \cite{Grunewald:ro5042}.

In typical experiments, the recorded diffraction patterns are processed by PR computer algorithms returning 
3D complex arrays, the modulus of which represents the electron density of the sample, while the phase is associated with
a component of the strain distribution inside the sample.  
Unlike other X-ray diffraction techniques, BCDI enables the study of isolated single particles, down to a few tens of 
nanometers in size \cite{MAXIV60nm, MIR20nm}, with spatial resolution on the electron density of the order of 10 nm \cite{cherukara_anisotropic_2018} 
and sensitivity to strain in the order of a few picometers \cite{Labat2015}.

BCDI experiments are carried out at beamline laboratories of synchrotron facilities or free-electron lasers (FELs) capable 
of delivering intense X-ray beams with exceptional coherence properties. Some examples are the ID01 beamline at the European 
Synchrotron Radiation Facility (ESRF) \cite{leake_nanodiffraction_2019}, the 34ID beamline at Advance Photon Source (APS) \cite{Pateras:yi5095}, the P10 beamline at PETRA III, 
NanoMax at MAX IV \cite{MAXIV60nm} or the CARNAÚBA beamline at SIRIUS \cite{Tolentino_2017}. 

\textbf{Chapter \ref*{chap:bcdi}} introduces the fundamental theoretical background of this technique and 
describes the practical conditions for a BCDI experiment at ID01.

The successful application of BCDI critically depends on computational algorithms that transform measured diffraction 
patterns into real-space reconstructions of particle shape and strain fields. From the early developments,
significant efforts have been devoted to improving the speed, robustness and reliability of these algorithms. These 
algorithms are usually available in Python or MATLAB based softwares, including Bonsu \cite{Newton2012Bonsu}, the widely 
used PyNX \cite{favre-nicolin_pynx_2020} with related toolkits \cite{Simonne2022Gwaihir, Atlan2023cdiutils} and the 
more recent SPRING \cite{Colombo2025SPRING}. 
This field has gained further momentum with the advent of machine learning (ML). In parallel, the recent upgrade of 
numerous synchrotron facilities to fourth-generation light sources — including MAX IV (2017), ESRF-EBS (2020), 
Sirius (2020), and APS (2025) — has dramatically increased coherent flux, boosting the potential of crystalline 
nano-imaging techniques such as BCDI and Bragg ptychography \cite{Li2022, leake_nanodiffraction_2019, PhysRevLett.121.256101, Chamard2015}. These advances also pose new 
challenges, particularly the need for faster and more efficient data processing pipelines capable of handling the rapidly 
growing volume of experimental data.
\newpage

This PhD project, as part of the ENGAGE doctoral 
program\footnote[1]{This project has been partly funded by the European Union's 
Horizon 2020 Research and Innovation Programme under the Marie Sklodowska-Curie COFUND scheme with grant agreement 
No. 101034267 and the European Research Council (ERC) under the European Union's Horizon 2020 Research and Innovation 
Programme (grant agreement No. 818823).}, and in collaboration with ID01 beamline and the Algorithms 
and Data Analysis group of the ESRF, was conceived in this context, with the goal of exploring how ML algorithms can address specific 
computational challenges in BCDI. \textbf{Chapter \ref{chap:phase_problem}} addresses the main aspects of the Phase 
Problem in BCDI and provides the necessary theoretical background on 
conventional phase retrieval algorithms and \textbf{Chapter \ref{chap:dl_theory}} introduces key ML concepts tailored 
to BCDI data analysis conducted throughout the PhD. Namely, two central problems are addressed in this work.

First, due to manufacturing constraints, detectors are often made of arrays of chips, butted together in a way that 
leads to data gaps within the diffraction patterns. \textbf{Chapter \ref{chap:inpainting}} discusses how convolutional neural networks 
(CNNs) can be employed to restore these missing data, thereby improving the reliability of the reconstructed images.

Second, X-ray detectors cannot directly measure the phase of the scattered wave-field, making its retrieval an inherently 
challenging computational task. \textbf{Chapters \ref{chap:phase_retrieval}} and \textbf{\ref{chap:AD_phase_retrieval}} 
present how ML-based approaches developed during this PhD can assist or complement conventional phase retrieval methods. 
