\section{Introduction}\label{chp:intro}

In this manuscript, the use of Deep Learning methods, and more in general of GPU accelerated optimizations, for the advance of 
the data analysis in Bragg Coherent Diffraction Imaging (BCDI) will be presented. However, before delving into the 
study’s developments, I would like to share with the reader a reflection that has taken shape over the course of this 
PhD, serving as a kind of preface. In particular, I have come to observe that, unlike other more fundamental 
scientific investigations, this work originates from the practical limitations of the technique in question. 
It is indeed because the detectors are unable to record flawless images due to gaps, or incapable of 
capturing phase information because its oscillations are too rapid—that one is compelled to manipulate the available 
data with sophisticated algorithms. And, as often happens in science, compensating for these technical shortcomings 
leads to the development of tools rooted in the most abstract realms of mathematics and information theory. How much 
missing information can one extract from a signal? How can it be extracted, and under what conditions? In which 
circumstances is it easier, and why? Thus, a fascinating world opens up not when we directly investigate the foundations 
of matter, but when we examine \textit{how} we go about investigating them. 

Although this manuscript is ultimately focused on the specific cases of BCDI gap inpainting and phase retrieval, 
I hope to convey at least a bit of the wonder and awe that comes from knowing that such applications draw their roots 
from far deeper, more general, complex, and abstract themes.

\subsection{PhD objectives and manuscript outline}

As will be presented in detail later, Bragg Coherent Diffraction Imaging (BCDI) is a powerful imaging technique performed 
at synchrotron and X-ray free-electron laser (XFEL) facilities. Thanks to its non-invasive nature and nanometer-scale 
spatial resolution, BCDI enables the 3D visualization, and investigation of the internal atomic structure of single-crystal nanoparticles 
with exceptional precision. This capability has proven valuable across diverse fields, including the study of strain 
distributions, defect populations, and particle morphologies under varying physico-chemical environments. Typical BCDI 
resolutions are on the order of 10 nm \cite{BCDI_review2024}.

Since its first demonstration in 2001 by Robinson et al. \cite{Robinson_gold_2001} for imaging gold nanoparticles, BCDI 
has been applied to technologically relevant materials in nanotechnology and electronics \cite{Favre-Nicolin_2010}, Li-ion 
and Na-ion batteries \cite{Singer2018, Serban2024}, catalysis \cite{atlan_imaging_2023}, and, more recently, biological 
materials \cite{Grunewald:ro5042}. Chapter \ref*{chap:bcdi} introduces the fundamental principles of this technique and 
describes the experimental setup available at the ID01 beamline of the ESRF.

The successful application of BCDI critically depends on computational algorithms that transform measured diffraction 
patterns into real-space reconstructions of particle shape and strain fields. Over the past two decades, significant 
efforts have been devoted to improving the robustness and reliability of these algorithms. This field has gained further 
momentum with the advent of machine learning (ML). In parallel, the recent upgrade of numerous synchrotron facilities 
to fourth-generation light sources — including MAX IV (Sweden, 2017), ESRF-EBS (France, 2020), and Sirius (Brazil, 2020) — 
has dramatically increased coherent flux, boosting the potential of crystalline nano-imaging techniques such as BCDI 
and Bragg ptychography \cite{Li2022, Leake:il5024}. These advances also pose new challenges, particularly the need 
for faster and more efficient data processing pipelines capable of handling the rapidly growing volume of experimental 
data.

This PhD project was conceived in this context, with the goal of exploring how ML algorithms can address specific 
computational challenges in BCDI. Chapter \ref{chap:phase_problem} provides the necessary theoretical background on 
conventional phase retrieval algorithms and introduces key ML concepts tailored to BCDI data analysis. Two central 
problems are addressed in this work.

First, X-ray detectors cannot directly measure the phase of the scattered wavefield, making its retrieval an inherently 
challenging computational task. Chapters \ref{chap:phase_retrieval} and \ref{chap:AD_phase_retrieval} present how ML-based 
approaches developed during this PhD can assist or complement conventional phase retrieval methods. Second, due to manufacturing 
constraints, detectors often contain grid-like regions that result in missing intensity data within the diffraction patterns. 
Chapter \ref{chap:inpainting} discusses how convolutional neural networks (CNNs) can be employed to restore these missing data, 
thereby improving the reliability of the reconstructed images.

% \subsection{Context}

% This PhD work was carried out entirely at the ESRF, bridging between the Algorithm and Data Analysis unit and the ID01 
% beamline which was also funding it. The