Le présent manuscrit traite du développement d'algorithmes d'apprentissage profond et d'optimisation accélérée sur 
GPU au moyen de la différenciation automatique pour l'analyse de données de la technique d'imagerie par diffraction 
cohérente de rayons X de Bragg (BCDI), menée par le doctorant au cours de trois années de recherche sur la ligne de lumière 
ID01 du synchrotron européen.

BCDI is a microscopy technique for the three-dimensional imaging of internal deformations and defects inside single 
crystals within 20 nm and 1$\mu$m size with resolution on the electron density of the order of 10 nm and resolution on 
the strain and deformations of the order of a few picometers. 
This technique does not employ optical lenses for the reconstruction of the images, hence its resolution is not affected 
by quality of the lenses. On the contrary, the reconstruction is entrusted to computer algorithms which solve the 
Phase Retrieval problem and yield the direct-space image of the particle and the corresponding strain field. 

The advancement of coherent diffraction imaging techniques made possible by the exceptional properties of x-ray sources 
available at 3rd generation synchrotron facilities has leapt forward recently with the increased brilliance and coherence 
guaranteed by the recent upgrade to 4th generation of the ESRF and other synchrotrons. At the same time, the tremendous 
progresses of deep learning for imaging suggest appealing solutions for the treatment of large data volumes expected 
with synchrotron upgrades. The exploration of the benefits of these data-driven approaches to the BCDI user community is 
the framework of this PhD thesis. 

The manuscript is therefore structured as follows: The first three Chapters \ref{chap:bcdi}-\ref{chap:phase_problem}-\ref{chap:dl_theory} 
are dedicated to the theoretical and practical background necessary for the understanding of the thesis results presented 
in the last three Chapters \ref{chap:inpainting}-\ref{chap:phase_retrieval}-\ref{chap:AD_phase_retrieval}. In particular: 
\begin{itemize}

  \item \textbf{Chapter \ref{chap:bcdi}} Presents a pedagogic derivation of the physical equations governing the coherent 
  x-ray diffraction from single small strained crystals. Assumptions and relative approximations are discussed to frame the validity 
  regimes of the equations. 

  \item \textbf{Chapter \ref{chap:phase_problem}} Discusses the Fourier Phase Problem.
  \item \textbf{Chapter \ref{chap:dl_theory}} 
  \item  \textbf{Chapter \ref{chap:inpainting}} 
  \item \textbf{Chapter \ref{chap:phase_retrieval}} 
  \item \textbf{Chapter \ref{chap:AD_phase_retrieval}} 
\end{itemize}