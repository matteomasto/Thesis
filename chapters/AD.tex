% In this last chapter an approach to the BCDI Phase Retrieval based on Automatic Differentiation will be discussed.
% It started from the necessity to 
% Unlke the DL model discussed above this method is iterative and

In this chapter a different approach to the BCDI phase retrieval will be presented. It originated from the need to resolve 
those cases in which neither standard alternating algorithms, nor the DL assisted PR can succeed to converge to a satisfactory 
reconstruction. The approach we developed differs from the alternating projections algorithms classically used for 
the Fourier PR, as it is formulated as minimization problem solved with gradient descent. The gradients however are computed 
through the efficient automatic differentiation (AD) enabled by graph-based differentiable programming packages like Tensorflow and 
PyTorch, accelerated on GPU. For this reason one could see the AD approach as unsupervised machine learning on a single training 
dataset.\\ 
The gradient-based optimization is fundamentally different from alternating projections. Here we could qualitatively say 
that if the latter switches between real and reciprocal space applying constraints in both, the former initializes a 
complex object and updates at each cycle its modulus and phase using the gradients, with respect to them, of the differences 
between the observed and calculated diffracted intensities. In this way, the knowledge on the particle can be implemented 
by initializing the object with some physical constraints or adding regularization terms that will drive the updates 
towards more reasonable solutions. \\

After mentioning the most relevant literature on AD phase retrieval for CDI, we will present our formulation of the problem 
and the results obtained on simulated and experimental BCDI patterns. 


\section{State of the Art}
\section{Model implementation}

\section{Results}
\subsection{Low-strain case}
\subsection{High-strain case}
\section{Conclusions}
