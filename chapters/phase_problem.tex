
% We enter now the 
This chapter is dedicated to the discussion of the \textit{Phase Problem} in BCDI and the main computational methods 
that are currently adopted to solve it. As anticipated briefly in the preface, the phase problem arises from a technical 
limitation. The fast oscillations of the electromagnetic fields in the X-ray regime induce detectors to only measure 
a time-averaged intensity (Eq.\ref{eq:poynting}) thus losing the phase information in the measurement. This problem 
has been known in the field of crystallography since the first measured diffraction patterns. It is interesting to acknowledge 
how a technical limitation can open an entire new field of research sinking its roots in the mathematics of complex functions, 
Fourier theory and inverse problems' optimization. The seek of the solution 
to the problem has fascinated (and still does) scientists for decades, contributing to an extensive production of works in 
literature. A subtlety is that while called \textit{Phase Problem} or \textit{Phase Retrieval}, it often aims at resolving 
the direct space object producing the diffraction pattern rather than reciprocal space phase that is lost. 

The first published studies on the retrieval of the date back to 1951 when Sayre in a comment \cite{Sayre_1952} to the paper by 
Shannon \textit{Communication in the presence of Noise} \cite{Shannon_1949} in which a condition on the sampling of the diffraction 
pattern was proposed for the restoration of the unit cell extent. Later in 1972 Gerchberg and Saxton \cite{gerchberg1972} developed an 
algorithm capable of inverting the diffraction pattern that is nowadays at the basis of currently used standard PR algorithms. 
However, proof of uniqueness of the solution arrived only later in 1979 and 1982 by Bruck and Sodin \cite{BruckSodin1979} and 
Bates \cite{Bates1982} In these works it was shown that for the 2D and 3D case the phase problem has unique solution in most cases, therefore 
conferring the mathematical solidity to the algorithm's results. A refined version of the Gerchberg - Saxton algorithm 
was proposed by J.R. Fienup in 1978 \cite{fienup_reconstruction_1978}
who named it Error Reduction (ER). In \cite{fienup_phase_1982}, published in 1982, the same author developed the Hybrid-Input Output (HIO) algorithm , 
able to outperform ER, and compared gradient-descent methods as well. In 1987 again Fienup showed the possibility of reconstructing 
\textit{complex-valued} objects if the constraints on the object support are ``tighter'' \cite{Fienup1987}. This result is 
particularly interesting for BCDI since, as we have seen in Eq.\ref{eq:fourie_relation}, the object to be retrieved is 
complex-valued. Based on the suggestion of Sayre in 1991 \cite{sayre1991direct} the works of Miao and coauthors from 1998 
opened the X-ray coherent diffraction imaging field addressing the phase retrieval combining the sampling proposed by Sayre and
iterative algorithms developed by Fienup \cite{Miao1998, Miao1999, Miao2000}.

For more detailed insights on the Phase Problem, the review published by Fannjiang and Strohmer in 2020 \cite{Fannjiang2020} is recommended. 
Mathematically the Phase Problem is classified as a non-linear inverse problem. The non-linearity caused by the 
modulo operation 

\section{Oversampling and detector distance}
\section{Alternating projections algorithms}
\subsection{Error Reduction (ER)}
\subsection{Hybrid Input-Output (HIO)}
\subsection{Relaxed Averaged Alternating Reflections (RAAR)}
\subsection{Shrinkwrap}

\section{Gradient descent based methods}
\subsection{Steepest descent}
\subsection{Conjugate Gradient Methods}

\section{Other methods}
\section{High strain and local minima}