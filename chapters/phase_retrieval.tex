

We enter now the core topic of the thesis. Most of the efforts during this PhD have been dedicated to the study of the Phase Problem 
for Bragg Coherent Diffraction Imaging using Deep Learning based approaches. Here we will discuss the main steps of this
journey, starting off from the analysis of the most relevant works in literature and concluding with our final version
of a DL model for highly strained particles. The latter has become the subject of an article, currently in preparation, 
entitled ``\textit{Phase Retrieval of Highly Strained Bragg Coherent Diffraction Patterns with Supervised Convolutional 
Neural Network}''. The process that led to the final version of the model will be unraveled, and particular attention
 will be given to elucidating the key steps and the critical issues encountered along the way. 

\section{State of the art}\label{chp:phasing_stateart}
In this paragraph we will focus on the state of the art for what concerns the Phase Retrieval of BCDI diffraction patterns with
deep-learning, tensor-computation and automatic differentiation methods. Conventional phase retrieval iterative algorithms 
are discussed in the introduction chapter as well as other approaches. \\
Given the relatively new development of neural networks and more specifically even more recent for BCDI phase retrieval we will try
to give a chronological broad overview over many of the main works in the literature pointing out strengths and weaknesses.
The first work pioneering the field is ``Real-time coherent diffraction inversion using deep generative networks'' published
by Cherukara \textit{et. al} in 2018 \cite{cherukara_real-time_2018}. The paper presents two CNNs for the phase retrieval of small ($32\times32$ pixels) 2D 
simulated BCDI patterns, one predicting the support and the other the phase. A U-Net like architecture with 
encoder-decoder was implemented, and the model was trained for just 10 epochs in a supervised fashion with a cross-entropy loss function (see Appendix).
The results show an excellent agreement between prediction and ground truth also in presence of relatively strong phases. 
The potential of this new approach for phase retrieval becomes immediately clear when considering the drastic reduction of
computational time and resources needed for the model inference. Once the model is trained the reconstruction can be obtained
within few milliseconds on a desktop machine. In 2020 Scheinker and Pokharel proposed another approach \cite{scheinker_adaptive_2020}
that employs a CNN model for 3D diffraction patterns. The fundamental difference is that the object's support is defined 
by its surface only, as it is assumed to be \textit{compact} and \textit{homogeneous} inside. Moreover, the surface is
parametrized by spherical harmonics and the DL model is trained to predict 28 of the first even coefficients of the spherical
harmonics. The model architecture is therefore essentially different since, while the encoder is just transposed to a 3D 
one, the decoder is replaced by a flattening and dense layer with 28 different classes as output. The model shows good performances
on both simulated and experimental data, marking the first DL-based approach capable of real 3D BCDI phase retrieval.
In the same year, Wu and coauthors \cite{Wu2021} opted for an architecture made of a single encoder and two identical decoders for the prediction of 
amplitude and phase of single crystals from the central slice of the BCDI pattern. They conducted the study on simulated 
data and tested it on one experimental case as well. What is evident from their work is the winning combination of DL prediction
and iterative refinement. The speed and generalization capabilities of the CNN allows for fast and good estimations of the 
object's support and phase. In addition, the precise and well established iterative methods can bring this initial guess to a 
more polished and accurate solution in fewer cycles than without DL prediction. This successful combined approach has been 
later adopted in other works, ours included. In 2021 two important works were published. First, Chan \textit{et al.} in 
\cite{chan_rapid_2021} extended the encoder/2-decoders architecture to the 3D case. In their work they first created a 
``physics-informed'' training set obtained building particles by clipping planes from a cubic FCC structure of atomic 
positions, relaxing them with LAMMPS software for molecular dynamics and computing the BCDI pattern around the (111) Bragg 
peak. The procedure is very similar to the one adopted by Lim \textit{et al.} in \cite{lim_convolutional_2021} and described
above in Section \ref{sec:dataset_creation3D}. Training the CNN on a restricted set of such created BCDI patterns biases 
the predictions towards physically meaningful particles. Although the authors managed to successfully test their model on
an experimental BCDI pattern, the small size ($32\times32\times32$ pixels) of the images accepted by the CNN was not yet 
enough for proper experimental use. It's with the work of Wu \textit{et al.} \cite{wu_three-dimensional_2021} published 
in the same year, which lifts the size to 64 pixel-sided cubes, that the model can be tested on several experimental cases. 
Their CNN model maintains the encoder/2-decoders architecture for a simultaneous prediction of the object's amplitude and phase 
and explores for the first time the unsupervised training for refinement as well. The authors claim that this approach is 
able to achieve better reconstruction quality with respect to current state-of-the-art iterative algorithms in use. 
The year after, Yao and coauthors published AutoPhaseNN \cite{yao_autophasenn_2022}, again an encoder/2-decoders architecture
that completely trained in an unsupervised manner. This approach is beneficial as it doesn't require datasets labeled with 
a ground truth, which means that experimental data can be directly used in the training set. Another advantage is that it 
overcomes the limitation of simulating an enough diverse population of samples, capable of constituting a comprehensive 
distribution of real cases. AutoPhaseNN is trained to predict an object the diffracted intensity of which matches the observed
one according to a normalized Mean Absolute Error metric. The model is shown to work on simulated data as well as on experimental 
data and once more the winning method lies in the combination of DL prediction and iterative refinement. 
AutoPhaseNN has marked a milestone in the BCDI data analysis, attaining 10X to 100X phase retrieval speed up with reduced efforts 
for the model training. 
Although of different nature, it is worth mentioning the work of Zhuang and coauthors \cite{Zhuang2022PracticalPR} in which 
two CNNs are used in the ``deep image prior'' (DIP) framework. DIP \cite{Ulyanov_2020} typically implies the use of a CNN for 
an enhanced representation of an image, often to solve inverse problems like super-resolution, denoising and inpainting. 
However, it differs from classical deep learning as there is no training dataset but a fit of the target problem exploiting
the parameters of the convolutional layers and the efficient gradient descent provided by the automatic differentiation. 
In their work, Zhuang \textit{et al.} formulated the more general far-field phase retrieval problem as an optimization problem 
and considered the phase symmetries that affect this class of solutions (see Introduction chapter). Their work employs two 
DIPs, one for the modulus and one for the phase, and successfully manages to reconstruct simulated objects even in presence 
of strong phases. 
A last interesting contribution is the work of Yu and \textit{et al.} \cite{yu_ultrafast_2024}. In this paper the authors
propose a DL model that computes complex convolutions, handling real and imaginary parts of the complex tensor in a single
passage through the convolutional block. Complex convolutional layers are claimed to be better at preserving the physical connection between real and imaginary parts  
inside the complex object. The model is used for the phase retrieval of experimental 2D diffraction patterns, for which an 
unsupervised refinement is used as well. \\
Before proceeding with our study, we summarize the main features of the mentioned works into a table, and we will comment about 
the different choices. 
 








\section{Reciprocal space phasing}\label{chp:phasing}
\section{Phase symmetries breaking}\label{chp:phasing}
\section{Model design}\label{chp:phasing}
\section{Results on 2D case}\label{chp:phasing}
\section{Results on 3D case}\label{chp:phasing}
\section{Refinement with iterative algorithms}\label{chp:phasing}
\section{Experimental results}\label{chp:phasing}

