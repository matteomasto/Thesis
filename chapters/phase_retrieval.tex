

We enter now the core topic of the thesis. Most of the efforts during this PhD have been dedicated to the study of the Phase Problem 
for Bragg Coherent Diffraction Imaging using Deep Learning based approaches. Here we will discuss the main steps of this
journey, starting off from the analysis of the most relevant works in literature and concluding with our final version
of a DL model for highly strained particles. The latter has become the subject of an article, currently in preparation, 
entitled ``\textit{Phase Retrieval of Highly Strained Bragg Coherent Diffraction Patterns with Supervised Convolutional 
Neural Network}''. The process that led to the final version of the model will be unraveled, and particular attention
 will be given to elucidating the key steps and the critical issues encountered along the way. 

\section{State of the art}\label{chp:phasing}
In this paragraph we will focus on the state of the art for what concerns the Phase Retrieval of BCDI diffraction patterns with
deep-learning, tensor-computation and automatic differentiation methods. Conventional phase retrieval iterative algorithms 
are discussed in the introduction chapter as well as other approaches. \\
Given the relatively new development of neural networks and more specifically even more recent for BCDI phase retrieval we will try
to give a chronological broad overview over many of the main works in the literature pointing out strengths and weaknesses.
The first

\section{Highly strained crystals}\label{chp:phasing}
\section{Reciprocal space phasing}\label{chp:phasing}
\section{Phase symmetries breaking}\label{chp:phasing}
\section{Model design}\label{chp:phasing}
\section{Results on 2D case}\label{chp:phasing}
\section{Results on 3D case}\label{chp:phasing}
\section{Refinement with iterative algorithms}\label{chp:phasing}
\section{Experimental results}\label{chp:phasing}

