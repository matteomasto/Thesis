
In this chapter the ``gap problem'' in the BCDI field is introduced as well as the state of the art measures that 
are taken to tackle it. It follows then a presentation of the various approaches that have been investigated using 
Deep Learning (DL) to conclude with the optimal one that is also discussed in the scientific paper named
 \textit{``Patching-based deep learning model for the Inpainting of Bragg Cohrent Diffraction patterns affected 
 by detectors' gaps''} (\url{https://doi.org/10.1107/S1600576724004163}) 

\section{State of the art}\label{chp:inpainting}
\section{Gap induced artifacts}\label{chp:phasing}
\section{Model design}\label{chp:phasing}
\section{Patching approach}\label{chp:phasing}
\section{Results in detector space}\label{chp:phasing}
\section{Results in real space}\label{chp:phasing}
\section{Fine tuning}\label{chp:phasing}
\section{Performances assessment}\label{chp:phasing}
