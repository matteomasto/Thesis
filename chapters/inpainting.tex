
In this chapter the ``detectors' gaps problem'' in Bragg Coherent Diffraction Imaging and our approach to solve it
using Deep Learning are discussed. The main state-of-the-art measures are presented briefly and
the topic of image inpainting with Deep Learning is introduced. The focus will then shift to our works that led
eventually to the optimal ``Patching-based'' approach that can also be found in the published paper entitled
 \textit{``Patching-based deep learning model for the Inpainting of Bragg Coherent Diffraction patterns affected 
 by detectors' gaps''} (\url{https://doi.org/10.1107/S1600576724004163}). The chapter is closed with some analyses 
 of the performances of the DL models in a variety of simulated and experimental cases.  

\section{The ``Gap Problem''}\label{sec:gaps}

At time of writing, standard BCDI experiments employ pixelated photon counting detectors to acquire the diffraction
patterns. These detectors can guarantee high spatial resolution, noise-free counting and fast read-out times. Two examples 
of these devices, currently used at the ID01 beamline are the MAXIPIX and EIGER detectors \cite{ponchut_maxipix_2011, Eiger_Johnson_2014}.
These detectors are often built by tiling together several sensing chips in order to cover a larger area, and are
typically bonded to an Application-Specific Integrated Circuit (ASIC) using bump bonding. 
This implies the presence, in the overall sensing region, of vertical and/or horizontal stripes that are not sensitive
to the impinging radiation. The width of these lines varies depending on the device but normally does not exceed the equivalent 
of a dozen of pixels. 
We have seen in chapter \ref{chp:intro} that during a BCDI scan the 2D images acquired by the detector are stacked to form
a 3D array. This leads these lines to become planes of missing signal in the dataset.
The problems arise when reconstructing the data affected by these gaps. In fact, these regions of non-physical zero intensity
deceive the Phase Retrieval algorithms inducing the presence of artifacts in the reconstructions\cite{carnis_towards_2019}
The typical artifacts caused by detector gaps are noticeable because of the presence of high-frequency oscillations in the 
amplitude and phase of the reconstructed object.

Moreover, these gaps tend to be problematic for the case of high-resolution BCDI, i.e. when the diffraction pattern
measurement extends to higher q-values, thus covering more than one sensing chip and necessarily crossing a gap region. 


\section{State of the art}\label{sec:InpStateArt}
\section{Model design}\label{sec:model}
\section{Patching approach}\label{sec:patching}
\section{Results in detector space}\label{sec:res_rec}
\section{Results in real space}\label{sec:res_real}
\section{Fine-tuning}\label{sec:finetuning}
\section{Performances assessment}\label{sec:performances}
