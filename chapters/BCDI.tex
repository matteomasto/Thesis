
% In this chapter a simplified and guided derivation of the x-ray scattering of a crystal structure under the so-called 
% kinematical approximation is presented. The formalism adopted here is the same as the one used by Jens Als-Nielsen and Des McMorrow 
% in \cite{alsnielsen_mcmorrow2011}. 
% Before starting, it is worth spending a few words on the different approaches that lead to the same final scattering equation through 
% different routes. The most complete, found in \cite{alsnielsen_mcmorrow2011} as well as \cite{ashcroft_mermin1976, guinier1994} 
% and many others, works with the scalar and vector potentials derived from Maxwell equations. The equations describing these potentials 
% can be solved for the specific case of an isolated point charge, (in our case the single electron inside the material that is 
% subjected to the perturbation of the impinging electromagnetic wave). The field radiated by the electron is therefore calculated and 
% subsequently extended to a collection of charges arranged in a periodic lattice, and the 

% The assumptions and related approximations made throughout this description are fundamentally
% \section{Theory of X-ray Coherent Diffraction in Bragg geometry}\label{chp:teory_bcdi}


\section{Coherent x-ray scattering from perfect crystalline structures}

In this chapter some basic theoretical insights about the BCDI technique are provided, with the aim to highlight the key 
concepts, assumptions and physical interpretations. More thorough descriptions can be found in papers, textbooks 
and PhD manuscripts. I will adopt the formalism of Als-Nielsen and McMorrow in \cite{alsnielsen_mcmorrow2011} but similar 
derivations and complementing observations can be found in \cite{guinier1994} as well as some more recent PhD thesis as.. 

\subsection{Foreword on typical BCDI assumptions and approximations}
In order to keep the dissertation short and targeted to the BCDI case, I will start considering some observations on the 
method that will lead to some preliminary assumptions and simplifications. First, the word ``Bragg'' suggests that 
crystalline specimens are involved. As discussed later in the text, Bragg's law applies to periodic structures, therefore 
we will limit our discussion to this specific case. 
The word ``Coherent'' implies that samples are probed with \textit{coherent} x-ray beams. This important property of 
synchrotron radiation will not be discussed in detail 

\subsection{Crystals}
\subsection{Laue condition and Bragg's Law} 
\subsection{Finite size crystals}
\subsection{Strained crystals}

\section{BCDI at ID01}\label{chp:phasing}
