
% In this chapter a simplified and guided derivation of the x-ray scattering of a crystal structure under the so-called 
% kinematical approximation is presented. The formalism adopted here is the same as the one used by Jens Als-Nielsen and Des McMorrow 
% in \cite{alsnielsen_mcmorrow2011}. 
% Before starting, it is worth spending a few words on the different approaches that lead to the same final scattering equation through 
% different routes. The most complete, found in \cite{alsnielsen_mcmorrow2011} as well as \cite{ashcroft_mermin1976, guinier1994} 
% and many others, works with the scalar and vector potentials derived from Maxwell equations. The equations describing these potentials 
% can be solved for the specific case of an isolated point charge, (in our case the single electron inside the material that is 
% subjected to the perturbation of the impinging electromagnetic wave). The field radiated by the electron is therefore calculated and 
% subsequently extended to a collection of charges arranged in a periodic lattice, and the 

% The assumptions and related approximations made throughout this description are fundamentally
% \section{Theory of X-ray Coherent Diffraction in Bragg geometry}\label{chp:teory_bcdi}


\section{Coherent x-ray scattering from perfect crystalline structures}

In this chapter some basic theoretical insights about the BCDI technique are provided, with the aim to highlight the key 
concepts, assumptions and physical interpretations. More thorough descriptions can be found in papers, textbooks 
and PhD manuscripts. I will adopt the formalism of Als-Nielsen and McMorrow in \cite{alsnielsen_mcmorrow2011} but similar 
derivations and complementing observations can be found in \cite{guinier1994, paganin2006coherent} as well as some more recent papers \cite{vartanyants2013coherentxraydiffractionimaging} 
and PhD thesis \cite{dupraz:tel-01285735, girard:tel-02906931}
, 

\subsection{Foreword on typical BCDI assumptions and approximations}
In order to keep the dissertation short and targeted to the BCDI case, I will start considering some observations on this
technique that will lead to some preliminary assumptions and simplifications. First, the word \textit{``Bragg''} suggests that 
crystalline specimens are involved. As discussed later in the text, Bragg's law applies to periodic structures, therefore 
we will limit our discussion to this specific case. \\
The word \textit{``Coherent''} implies that samples are probed with coherent beams (in our case x-rays). This fundamental property of 
synchrotron radiation will be briefly discussed later on. For the moment, this ingredient enables us to express the 
probing radiation with plane electromagnetic waves. \\
The word \textit{``Diffraction''} refers to the type of mechanism describing the interaction between the x-rays and the 
samples. Paraphrasing \cite{guinier1994} at page 4, this mechanism can be divided into two main phenomena, namely (i) 
the scattering of the radiation by each individual atom in the sample and (ii) the interference between the waves scattered 
by these atoms. The interference mechanism, in turn, is enabled because these scattered waves are coherent with the incident 
radiation and therefore between themselves. In other words, the information of each scatterer is shared with the other scatterers 
as the diffracted waves ``talk to each other''. The complete mathematical description of these two phenomena without approximations 
is prohibitive, hence some simplifications usually adopted: 
\begin{itemize}
    \item \textbf{No refraction, no absorption}: Scattering is the only mechanism considered. Because of their short wavelengths
    (0.5 - 2.5 \AA), x-rays are practically never deviated by refraction. Moreover, we assume to always operate at energies that 
    are far from absorption edges of the probed materials, thus neglecting any absorption effect. 
    \item \textbf{Elastic scattering}: The interaction between the incoming x-ray and the atom is elastic, meaning that 
    no energy nor momentum is transferred to the atom, which bounces off instead the photons with unaltered energy and 
    momentum. This is again necessary for the scattered waves to interfere, as any difference in wavelength would 
    prevent any coherent interaction. This description, also called Thomson scattering, considers the interaction with a free charge, 
    and it also shows that cross-section of the scattering of electrons is much higher than the one of protons. 
    For this reason, only the scattering from electrons is considered. 
    \item \textbf{Weak diffraction (Born approximation)}: This assumption implies that each scattered wave does not interact 
    further with the sample, therefore neglecting any possible multiple scattering event. The consequence of this assumption 
    is that the overall diffracted wave can be approximated by the linear superposition of the contributions of each scattering site. 
    This approximation, in crystallography, is called \textit{kinematical approximation}. 
    Dealing with crystalline samples, this assumption breaks for relatively thick samples ( $ > 1 \mu m $) in which the 
    light travels through the sample for longer distances before exiting, therefore bouncing off several atoms. 
    Diffraction of larger samples requires more complex theory of the so-called \textit{dynamical regime}.
    However, in our case, the size of the typical samples studied with BCDI hardly exceeds $ 1 \mu m $ size, making the 
    kinematical approximation suitable. 
    \item \textbf{Far-field approximation} The last important 

\end{itemize}

The last word \textit{``Imaging''} tells us that the format of the data is by nature, multidimensional (2D - 3D). 

\subsection{Crystals}
\subsection{Laue condition and Bragg's Law} 
\subsection{Finite size crystals}
\subsection{Strained crystals}

\section{BCDI at ID01}\label{chp:phasing}
