\chapter{Appendix}\label{chp:appendix}

\section{Stirling}
We can approximate $N!$ as:
\begin{equation}
    N! \simeq \sqrt{2 \pi N} \left( \frac{N}{e} \right)^N
\end{equation}


\begin{equation}
    ln(N!) \simeq ln \left( \sqrt{2 \pi N} \right) + ln \left( \left( \frac{N}{e} \right)^N \right)
\end{equation}

\begin{equation}
    ln(N!) \simeq \frac{1}{2} ln(2 \pi N) + N ln(N) - N
\end{equation}

For large N, the first term is negligible.

\begin{equation}
    ln(N!) \simeq N \left( ln(N) \right) - N
\end{equation}








\section{Classical distinguishable particles}


For a general system with $n$ different possible states for each particle the macrostate is defined by the distribution of number of particles, $N_i$, in each state $i$. For a given distribution/macrostate, or set of $N_i$(s) ($\{N_i\}$), we can count then number of different microstates that correspond to this macrostate:


\begin{equation}
    w_{ \{ N_i\}} = \frac{N!}{\prod_{i=1}^{n} N_i!} 
    \label{Aeq:thermo_prob}
\end{equation} 

Now to find the correct density operator we must maximize the entropy while respecting the natural constraints of the system. 
This can be expressed as the maximization of a function with constraints and is done using Lagrange multipliers. We can write 

\begin{equation}
    \mathcal{L}(w, \lambda) = ln(w) + \lambda_1 \sum_{i} N_i + \lambda_2 \sum_{i} E_i N_i
\end{equation}

Where our constraints are:

\begin{equation}
    \label{Aeq:constr}
    \sum_{i} N_i  = N \quad \text{,} \quad  \sum_{i} E_i N_i = \langle E \rangle
\end{equation}

Here we take the example where we only known the average energy of the system.
In the Lagrange multiplier method the first term in our equation is the function we want to maximize followed by the constraints. Each constraint is multiplied by a term known as the Lagrange multipliers ($\lambda_1 , \lambda_2$). We can further develop this by substituting in equation \ref{eq:thermo_prob}.

\begin{equation}
    \mathcal{L}(w, \lambda) = ln\left( \frac{N!}{\prod_{i=1}^{n} N_i!}  \right) + \lambda_1 \sum_{i} N_i + \lambda_2 \sum_{i} E_i N_i
\end{equation}

We can simplify using Stirling's approximation (see Appendix \ref{chp:appendix}), and using the properties of the product of a series and the natural log.


\begin{equation}
    \mathcal{L}(w, \lambda) =   ln\left( N! \right) - \sum_{i = 1}^{n} N_i ln \left( N_i \right )  + \sum_{i} N_i  + \lambda_1 \sum_{i} N_i + \lambda_2 \sum_{i} E_i N_i
\end{equation}


We must now take the derivative with respect to $N_j$ to search for the stationary points. A different index is used to not confuse the derivation index and the summation index. We note however that for terms in the sum with $i \neq j$ the derivative is zero. The first term $ln\left( N! \right)$ is also constant when derived with respect to $N_j$. Thus we are left with

\begin{equation}
    \frac{\partial f}{\partial N_i} = - ln(N_i) + \lambda_1 + \lambda_2 E_i = 0
\end{equation}

\begin{equation}
    \label{Aeq:N_i}
    N_i = e^{\lambda_1 + \lambda_2 E_i}
\end{equation}

By re-inserting this into our constraints, we find

\begin{equation}
   e^{\lambda_1} \sum_{i} E_i e^{\lambda_2 E_i} = \langle E \rangle \quad \text{,} \quad    e^{\lambda_1}\sum_{i} e^{\lambda_2 E_i}  = N
\end{equation}

We now make the strategic choice to change the variable names:

\begin{equation}
    \lambda_1 = \alpha \quad \text{,} \quad \lambda_2 = - \beta
\end{equation}

We also define $\mathcal{Z}$ known as the \textit{partition function}:

\begin{equation}
    \mathcal{Z} = \sum_{i} e^{- \beta E_i}
\end{equation}

Thus, we are left with:

\begin{equation}
    \label{Aeq:e_alpha_z}
    e^{\alpha} \mathcal{Z} = N 
\end{equation}

From equation \ref{Aeq:N_i}

\begin{equation}
    N_i = e^{\alpha - \beta E_i}
\end{equation}
 
Substituting in from equation \ref{Aeq:e_alpha_z}

\begin{equation}
    \label{Aeq:N_i_overZ}
    N_i = \frac{N}{\mathcal{Z}} e^{-\beta E_i}
\end{equation}

This equation can be read as the number of particles in a given (quantum) state, with energy $E_i$, for a given ensemble microstate, and is known as the \textit{Boltzmann distribution}.

This can be rewritten as:

\begin{equation}
    \frac{N_i}{N} = \frac{1}{\mathcal{Z}} e^{-\beta E_i}
\end{equation}

The left-hand side can be interpreted as the probability distribution of particles in different states. This can be re-written in terms of operators as:

\begin{equation}
    \label{Aeq:dens_z}
    \hat{\rho} =  \frac{1}{\mathcal{Z}} e^{-\beta \hat{H}}
\end{equation}

Recalling equation \ref{eq:trace} we see that for the hamiltonian operator of the system the average energy can be expressed as:

\begin{equation}
    \langle \hat{E} \rangle = Tr \left( \hat{\rho} \hat{H} \right)
\end{equation}

The first constraint from equation \ref{eq:constr} implies the normalization of the density operator and can be re-expressed as

\begin{equation}
    Tr(\hat{\rho}) = 1
\end{equation}

Combining equations \ref{eq:dens_z} with the normalization condition gives

\begin{equation}
    \mathcal{Z} = Tr(e^{-\beta \hat{H}})
\end{equation}

The preceding developments are for a system where the energy of the system is known only as an average, and the number of particles is fixed. This system is known as the \textit{canonical ensemble}\footnote[1]{Also the volume is fixed in the canonical ensemble. Not sure why we didn't take this into account explicitly}. If we substitute equation \ref{Aeq:N_i} into the second constraint from equation \ref{Aq:constr} we find

\begin{equation}
    \langle \hat{E} \rangle = \frac{N}{\mathcal{Z}} \sum_{i} E_i e^{-\beta E_i}
\end{equation}

Looking at the definition of the partition function this can be rewritten as 
\begin{equation}
    \langle \hat{E} \rangle = - \frac{N}{\mathcal{Z}} \frac{\partial \mathcal{Z}}{\partial \beta}
\end{equation}

or equivalently

\begin{equation}
    \langle \hat{E} \rangle = - N \frac{\partial ln \left( \mathcal{Z} \right)}{\partial \beta}
\end{equation}

Going back to the entropy as defined in equation \ref{eq:entropy} and using the Stirling approximation as used in the Lagrange multipliers

\begin{equation}
    \frac{S}{k_B} = N ln(N) - \sum_{i} N_i ln(N_i) 
\end{equation}

Substituting in equation \ref{Aeq:N_i} and equation \ref{Aeq:e_alpha_z} \footnote[1]{Should be careful I am following exactly the development of Garanin}

\begin{equation}
    \frac{S}{k_B} = N ln(N) - \sum_{i} N_i (\alpha - \beta E_i)  = N ln(\mathcal{Z}) + \beta \langle E \rangle
\end{equation}

Now examining the variation of the entropy with respect to other parameters, we calculate its differential

\begin{equation}
    dS = \frac{dS}{d\beta} = \left( N \frac{\partial ln(\mathcal{Z})}{\partial \beta} + \langle E \rangle \frac{\partial \beta}{\partial \beta} + \beta \frac{d \langle E \rangle}{\partial \beta} \right) d\beta= \left(-\langle E \rangle + \langle E \rangle + \beta \frac{d \langle E \rangle} {d\beta} \right) d\beta
\end{equation}

\begin{equation}
    dS = \beta d\langle E \rangle
\end{equation}

For a system with a a large number of particles in thermodynamic equilibrium the average energy of the system is the internal energy $U$. This allows us to exploit the fundamental thermodynamic principal.

\begin{equation}
    dU = TdS - PdV
\end{equation}

In the canonical ensemble the volume is fixed. This allows us to deduce 

\begin{equation}
    \beta = \frac{1}{k_B T}
\end{equation}



\section{Positive E to current flow}

If we have a positive $E$ as defined in equation \ref{eq:elec_pot}. Then,

\begin{equation}
    E > 0 \implies \Phi_R > \Phi_L
\end{equation}

\begin{equation}
    \mu_\textit{e}^- = \mu_{chem} + q\Phi
\end{equation}

$q$ is the charge of an electron

Since the electrons are both in the same material
\begin{equation}
    \mu_{chem, \textit{e}^{-}_{R}} = \mu_{chem, \textit{e}^{-}_{L}}
\end{equation}

Since any electrical potential measurement between two electrodes measures the difference in electrochemical potential than the difference in chemical potentials for the same material is zero. This is also a requirement to use the definition of cell voltage in equation \ref{eq:elec_pot}.

\begin{equation}
    \mu_{\textit{e}^-_R} = -\textit{e} \Phi_R \quad \text{,} \quad \mu_{\textit{e}^-_L} = -\textit{e} \Phi_L
\end{equation}

\begin{equation}
    \Phi_R > \Phi_L \implies \mu_L > \mu_R
\end{equation}

For a system with temperature and pressure fixed (isobaric-isothermal ensemble) the system will respond to this dis-equilibrium with the only free parameters it can: $N_L$ and $N_R$.

So, $N_L$ will decrease and $N_R$ will increase. This is a flow of particles (electrons) from left to right. Thus we will measure a positive electrical current from right to left.